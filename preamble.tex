\usepackage{float}
\usepackage[utf8]{inputenc}
\usepackage[a4paper,footskip=0.25in]{geometry}
\usepackage{xcolor}
\usepackage{amsmath}
\usepackage{amsthm}
\usepackage{amssymb}
\usepackage{amsfonts}
\usepackage{mathtools}
\usepackage{hyperref}
\usepackage{ dsfont }
\usepackage[most]{tcolorbox}
\usepackage{tikz}
\usepackage{xcolor}
\usepackage{adjustbox}
\usepackage{mathrsfs}
\usepackage[chapter]{algorithm}
\usepackage{algpseudocode}
\usepackage{marginnote}
\usepackage{xhfill}
\usepackage{mparhack}
\usepackage{pagecolor}
\usepackage{titlesec}
\usepackage{sectsty}
\usepackage{tocloft}
\usepackage{lastpage}
\usepackage{fancyhdr}
\usepackage{tocloft}
\usepackage{soul}
\usepackage{etoolbox}
\usepackage{xstring}

\definecolor{mygray}{RGB}{239,240,241}
\definecolor{mycolor}{RGB}{146,0,88}
\usetikzlibrary{babel}
\usetikzlibrary{cd}
\usetikzlibrary{positioning}
\usetikzlibrary{calc}
\usetikzlibrary{arrows.meta}

\geometry{
	left=10mm, % left margin
	textwidth=150mm, % main text block
	marginparsep=5mm, % gutter between main text block and margin notes
	marginparwidth=40mm, % width of margin notes
  bmargin=2cm % height of foot note space
}
\graphicspath{ {./figures/} }


\setlength{\cftchapnumwidth}{2.25cm}
\renewcommand{\cftchappresnum}{\chaptername\ }


\pagestyle{fancy}

\renewcommand{\headrule}{\color{red}\hrulefill}

\newlength\FHoffset
\setlength\FHoffset{1cm}

\addtolength\headwidth{2\FHoffset}

\fancyheadoffset{\FHoffset}

% these lengths will control the headrule trimming to the left and right 
\newlength\FHleft
\newlength\FHright

% here the trimmings are controlled by the user
\setlength\FHleft{1cm}
\setlength\FHright{-3.25cm}

% The new definition of headrule that will take into account the trimming(s)
\newbox\FHline
\setbox\FHline=\hbox{\hsize=\paperwidth%
  \hspace*{\FHleft}%
  \rule{\dimexpr\headwidth-\FHleft-\FHright\relax}{\headrulewidth}\hspace*{\FHright}%
}
\renewcommand\headrule{\color{mycolor}\vskip-.7\baselineskip\copy\FHline}

\renewcommand{\chaptermark}[1]{\markboth{#1}{#1}}
\fancyhead[R]{}
\fancyhead[L]{}
\fancyhead[C]{\hspace{3.25cm}\color{mycolor}\chaptername\ \thechapter\ --\ \leftmark}
\fancyfoot[L]{Page \thepage \hspace{1pt} of \pageref{LastPage}}
\fancyfoot[C]{}

\fancypagestyle{plain}{%
  \fancyhf{}%
  \fancyfoot[L]{Page \thepage \hspace{1pt} of \pageref{LastPage}}
  \renewcommand{\headrulewidth}{0pt}
  \renewcommand{\footrulewidth}{0pt}
  
}

\def\C{\mathbb{C}}
\def\R{\mathbb{R}}
\def\N{\mathbb{N}}
\def\Q{\mathbb{Q}}
\def\Z{\mathbb{Z}}
\def\cA{\mathcal{A}}
\def\cM{\mathcal{M}}
\def\cE{\mathcal{E}}
\def\cF{\mathcal{F}}
\def\cS{\mathcal{S}}
\def\cC{\mathcal{C}}
\def\cL{\mathcal{L}}
\def\bP{\mathbb{P}}
\def\bE{\mathbb{E}}
\def\Mloc{\mathcal{M}_{\text{loc}}}
\def\as{\text{ a.s. }}
\def\cov{\text{Cov}}
\def\sgn{\text{sgn}}


\newtheorem{theorem}[algorithm]{Theorem}
\newtheorem{corollary}[algorithm]{Corollary}
\newtheorem{lemma}[algorithm]{Lemma}
\newtheorem{proposition}[algorithm]{Proposition}
\newtheorem{definition}[algorithm]{Definition}
\newtheorem*{*definition}{Definition}
\newtheorem*{remark}{Remark}
\newtheorem*{aremark}{Added remark}
\newtheorem{example}[algorithm]{Example}
\newtheorem*{adefinition}{Added definition}
\newtheorem*{aexample}{Example}
\raggedright


\hypersetup{
    colorlinks=true, %set true if you want colored links
    linktoc=all,     %set to all if you want both sections and subsections linked
    linkcolor=mycolor,  %choose some color if you want links to stand out
    linkbordercolor=mycolor,
    urlcolor=mycolor
}

\chapterfont{\color{blue}}

\titleformat{\chapter}[display]
  {\normalfont\bfseries\flushleft\color{mycolor}}{\Huge\chaptertitlename\ \thechapter:}{0pt}{\Huge}
\titleformat{\section}[hang]
  {\color{mycolor}}{}{0pt}{}

\sectionfont{\color{mycolor}}

\subsectionfont{\color{mycolor}}


\renewcommand{\cftchapfont}{\color{mycolor}\large\bfseries}
\renewcommand{\cftsecfont}{\color{mycolor}}
\renewcommand{\cftsubsecfont}{\color{mycolor}}
\renewcommand{\cftsubsubsecfont}{\color{mycolor}}

\renewcommand{\cftchapleader}{\color{mycolor}     \cftdotfill{\cftsecdotsep}}
\renewcommand{\cftsecleader}{\color{mycolor}     \cftdotfill{\cftsecdotsep}}
\renewcommand{\cftsubsecleader}{\color{mycolor} \cftdotfill{\cftsecdotsep}}
\renewcommand{\cftsubsubsecleader}{\color{mycolor} \cftdotfill{\cftsecdotsep}}

\def\lecturelist{}

\newcommand{\beginlecture}[2]{\hypertarget{lec#1}{}\listxadd{\lecturelist}{lec#1}\marginnote{Start of lecture #1 (#2)}}
\newcommand{\dhighlight}[1]{{\color{mycolor}{\underline{{\textbf{#1}}}}}}
\newcommand{\highlight}[1]{{\color{mycolor}{\textbf{#1}}}}

\newcommand{\includeproblem}[2]{
\begin{tcolorbox}[enhanced,breakable,
	title=Problem #1,frame style={color=mycolor}]
	\input{#2}
\end{tcolorbox}}
\newcommand{\includesolution}[2]{
\begin{tcolorbox}[enhanced,breakable,
	title=Solution #1,frame style={color=mycolor}]
	\input{#2}
\end{tcolorbox}}

\renewcommand{\labelitemi}{$\textcolor{mycolor}{\bullet}$}
\renewcommand{\labelitemii}{$\textcolor{mycolor}{\cdot}$}
\renewcommand{\labelitemiii}{$\textcolor{mycolor}{\diamond}$}
\renewcommand{\labelitemiv}{$\textcolor{mycolor}{\ast}$}


\tcolorboxenvironment{corollary}{
  colback=blue!5!white,
  boxrule=0pt,
  boxsep=1pt,
  left=2pt,right=2pt,top=2pt,bottom=2pt,
  oversize=2pt,
  sharp corners,
  before skip=\topsep,
  after skip=\topsep,
  breakable
}

\tcolorboxenvironment{theorem}{
  colback=blue!5!white,
  boxrule=0pt,
  boxsep=1pt,
  left=2pt,right=2pt,top=2pt,bottom=2pt,
  oversize=2pt,
  sharp corners,
  before skip=\topsep,
  after skip=\topsep,
  breakable
}

\tcolorboxenvironment{lemma}{
  colback=blue!5!white,
  boxrule=0pt,
  boxsep=1pt,
  left=2pt,right=2pt,top=2pt,bottom=2pt,
  oversize=2pt,
  sharp corners,
  before skip=\topsep,
  after skip=\topsep,
  breakable
}

\tcolorboxenvironment{proposition}{
  colback=blue!5!white,
  boxrule=0pt,
  boxsep=1pt,
  left=2pt,right=2pt,top=2pt,bottom=2pt,
  oversize=2pt,
  sharp corners,
  before skip=\topsep,
  after skip=\topsep,
  breakable
}

\tcolorboxenvironment{*definition}{
  colback=blue!5!white,
  boxrule=0pt,
  boxsep=1pt,
  left=2pt,right=2pt,top=2pt,bottom=2pt,
  oversize=2pt,
  sharp corners,
  before skip=\topsep,
  after skip=\topsep,
  breakable
}

\tcolorboxenvironment{remark}{
  colback=red!5!white,
  boxrule=0pt,
  boxsep=1pt,
  left=2pt,right=2pt,top=2pt,bottom=2pt,
  oversize=2pt,
  sharp corners,
  before skip=\topsep,
  after skip=\topsep,
  breakable
}

\tcolorboxenvironment{aremark}{
  colback=red!5!white,
  boxrule=0pt,
  boxsep=1pt,
  left=2pt,right=2pt,top=2pt,bottom=2pt,
  oversize=2pt,
  sharp corners,
  before skip=\topsep,
  after skip=\topsep,
  breakable
}

\tcolorboxenvironment{example}{
  colback=red!5!white,
  boxrule=0pt,
  boxsep=1pt,
  left=2pt,right=2pt,top=2pt,bottom=2pt,
  oversize=2pt,
  sharp corners,
  before skip=\topsep,
  after skip=\topsep,
  breakable
}

\tcolorboxenvironment{adefinition}{
  colback=blue!5!white,
  boxrule=0pt,
  boxsep=1pt,
  left=2pt,right=2pt,top=2pt,bottom=2pt,
  oversize=2pt,
  sharp corners,
  before skip=\topsep,
  after skip=\topsep,
  breakable
}

\tcolorboxenvironment{aexample}{
  colback=red!5!white,
  boxrule=0pt,
  boxsep=1pt,
  left=2pt,right=2pt,top=2pt,bottom=2pt,
  oversize=2pt,
  sharp corners,
  before skip=\topsep,
  after skip=\topsep,
  breakable
}

\tcolorboxenvironment{definition}{
  colback=green!5!white,
  boxrule=0pt,
  boxsep=1pt,
  left=2pt,right=2pt,top=2pt,bottom=2pt,
  oversize=2pt,
  sharp corners,
  before skip=\topsep,
  after skip=\topsep,
  breakable
}

\tcolorboxenvironment{algorithm}{
  colback=mygray,
  boxrule=0pt,
  boxsep=1pt,
  oversize=2pt,
  left=2pt,right=2pt,top=-12pt,bottom=0pt,
  sharp corners,
  breakable
}