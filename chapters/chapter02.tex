\chapter{SDE techniques}

\dhighlight{Goal:} Study process by changing the measure.

E.g.: $X_t=B_t$. $\underbrace{\text{Condition }X_t\geq 0\forall t\geq 0}_{\eqqcolon Y_t}$.

% Figure


\section{Girsanov theorem}

Given $(\Omega,\cF,(\cF_t)_{t\geq 0})$ and two measures $\bP,\Q$\marginnote{The notes of Eberle switches the roles of $\bP,\Q$!}

Assume $\bP\ll \Q$ and $\Q\ll\bP$
and let $H=\frac{d\Q}{d\bP}$
\[\implies Z_t\coloneqq \bE_\bP(H\mid \cF_t)=\frac{d\Q}{d\bP}\mid_{\cF_t}\]
is a martingale.

From last semester: $\forall Y\in L^1(\Q)\cap L^1(\bP),\cF_t$ measurable: %TODO: Check
\begin{equation*}
    \bE_\Q(Y\mid \cF_s)=\frac{bE_{\bP}(Y\cdot Z_t\Big\vert\cF_s)}{Z_s}\forall s<t
\end{equation*}
If $Z>0$ is $\Mloc,\exists L \in\Mloc$ s.t.:
\[Z_t=e^{L_t-\frac{1}{2}\langle L\rangle_t}\to L_t = \ln(Z_0)+\int_0^t \frac{d Z_s}{Z_s}\] 

There might be a problem, because $\ln(Z_0)$ might not be integrable, and therefore not a local martingale!

\begin{theorem}[Girsanov]\label{thm:2.1:Girsanov}
    Assume $Z>0$ is a martingale. If $M$ is a local martingale w.r.t. $\bP$, then 
    \[\tilde{M}_t\coloneqq M_t-\langle M,L\rangle_t\] is a local martingale w.r.t. $\Q$
    and \[\langle M\rangle_t=\langle \tilde{M}\rangle_t.\]
    Moreover, if $M$ is a BM w.r.t. to $\bP$, then $\tilde{M}$ is a BM w.r.t. $\Q$.
\end{theorem}

\begin{remark}
    In applications, given $\bP,(Z_t)_{t\geq 0}$
    a positive continuous martingale, 
    define $\Q$ on $\cF_\infty=\bigcup_{t\geq 0}\cF_t$ s.t. 
    \[\frac{d\Q}{d\bP}\Big\vert_{\cF_t}=Z_t\]
    $Z$ is uniformly integrable $\iff \Q\ll\bP$. 
    \highlight{Problem:} In applications, 
    $Z$ is not necessarily uniformly integrable.

    $\implies$ restrict to $[0,T]\implies$ all fine.

    $\Q\to\Q_T$ as in the last semester.
\end{remark}


\begin{aexample}
    Let $\gamma\in\R^d,(B_t)_{t\geq 0}$ standard BM. 

    Let $L_t\coloneqq \gamma\cdot B_t$ and $Z_t=\exp(L_t-\frac{1}{2}\langle L\rangle_t)=\exp(\gamma \cdot B_t-\frac{1}{2}|\gamma|^2t)$
    
\end{aexample}

\begin{remark}
    $\lim_{M\to\infty}\sup_{t\geq 0}\bE(|Z_t|1_{|Z_t|>M})\neq 0$.

    Define $\Q$ on $\cF_\infty$ s.t. $\tilde{B}_t^k=B_t^k-\langle L,B^k\rangle=B_t^k-\gamma^kt$\marginnote{Construct $Q$ via $Z$}
    is a BM with drift. Show: $Q\not\ll\bP$.

    \[A=\left\{\lim_{t\to\infty}\frac{\tilde{B}_t+\gamma t}{t}=\gamma\right\}\in\cF_\infty\]
    but: $\tilde{B}$ is a BM w.r.t. $\Q\implies \Q(A)=1$, $\tilde{B}$ is a BM with drift $-\gamma$ w.r.t 
    $\bP(A)=0$.
\end{remark}

\beginlecture{06}{30.04.24}

\subsection{Drift transformation of SDE}

We once again start with

\[\star=\begin{cases}
    dX_t&=b(t,X_t)\\
    X_0&=x_0
\end{cases}\]
with drift $b$ continuous.

\highlight{Goal:} Get a weak solution of $\star$.

Let $(X_t)_{t\geq 0}$ be a BM in $(\Omega,\cF(\cF_t)_{t\geq 0},\bP),X_0=x_0$.

\highlight{Assume:} $Z_t\coloneqq \exp\left(\underbrace{\int_0^t b(s,X_s)dX_s}_{=L_t}-\frac{1}{2}\int_0^t |b(s,X_s)|^2ds\right)$ 
is a martingale w.r.t. $\bP$.

\begin{remark}
    Thr assumption holds if 
    \[|b(t,x)|\leq C(1+\Vert x\Vert)\text{ for some }C.\]
\end{remark}

$\implies \bE_\bP(Z_t)=1\forall t\geq 0$

By Girsanov
\begin{align*}
    \tilde{X}_t=X_t-\langle L,X\rangle_t
\end{align*}
is a $\Q$-BM.

But \begin{align*}
    d\langle L,X\rangle_t&=b\cdot dX_t\cdot dX_t=b\cdot dt\\
    \implies \tilde{X}_t &= X_t -\int_0^tb(s,X_s)ds\\
    \text{w.r.t. }\Q:& dX_t = b(t,X_t)dt+\underbrace{d\tilde{B}_t}_{d\tilde{X}_t}
\end{align*}

Generalization: Start with 

\[
\star\star dX_t = b(t,X_t)dt+\sigma(t,X_t)dB_t    
\]

where $X,B$ are $d$-dimensional.

\begin{proposition}\label{prop:2.2}\marginnote{In practice this is surprisingly useful in practice!}
    Assume $(X,B)$ is a weak solution of $\star\star$ on $(\Omega,\cF,(\cF_t)_{t\geq 0},\bP)$.
    If \[Z_t=\exp\left(\int_0^t c(s,X_s)dB_s-\frac{1}{2}\int_0^t c(s,X_s)^2ds\right)\]
    is a martingale \marginnote{this is an implicit condition for $c$}
    w.r.t. $\bP$, $\Q\ll\bP$ and $\cF_t$ s.t. $Z_t=\frac{d\Q_t}{d\bP}\text{Vert}_{\cF_t}$.

    Then $(X_t)_{t\geq 0}$ on $(\Omega,\cF,(\cF_t)_{t\geq 0},\Q)$ is a weak soltuion of 
    \[dX_t = [b(t,X_t)+\sigma(t,X_t)c(t,X_t)dt]+\sigma(t,X_t)d\tilde{B}_t\]
    where $\tilde{B}$ is a d-dim BM.
\end{proposition}

\begin{proof}
    \begin{align*}
        d\langle L,B\rangle_t&\stackrel{L_t=\int_0^t c(s,X_s)dB_s}{=}c(t,X_t)dt\\
        \implies& \tilde{B}_t\coloneqq B_t-\langle L,B\rangle_t \text{ is a }\Q \text{-BM}\\
        \implies & dB_t = c(t,X_t)dt+d\tilde{B}_t
    \end{align*}

    From $\star\star$: $dX_t=b(t,X_t)dt+\sigma(t,X_t)c(t,X_t)dt+\sigma(t,X_t)d\tilde{B}_t=\star\star\star$
\end{proof}

\begin{equation*}%TODO: FIX
    \begin{array}{|l|l|l|}\hline
        \text{Measure} & \text{SDE } & \text{Generators}\\
        \hline
        \bP& dX_t b\cdot dt +\sigma \cdot dB_t& \mathcal{L}\sum_k b_k\frac{\partial}{\partial x_k}+\frac{1}{2}\sum_{k,l}(\sigma\sigma^\intercal)_{k,l}\frac{\partial^2}{\partial x_k\partial x_l}\\
        \hline
        \Q&[b+\sigma\cdot c] dt+\sigma d\tilde{B}_t & \tilde{\cL}=\cL+\sum_{k}\sum_l \sigma_{k,l}c_l\frac{\partial}{\partial x_k}=\cL + c\sigma^\intercal \nabla\\
        \hline
    \end{array}
\end{equation*}

$t\leq T$ for $\Q_t$. 

\section{Doob-h transform}
%TODO: Images
\begin{enumerate}\marginnote{Essentially three cases in 1d}
    \item From $B_t\to B_t$ conditioned on $\{B_1=0\}$ (measure zero set)
    \item From $B_t\to B_t$ conditioned on $\{B_t\geq 0\forall t\geq 0\}$ (measure zero set)
    \item From $B_t\to B_t$ conditioned on $\{B_t\geq 0\forall t\in [0,1],B_1=0\}$
\end{enumerate}

Let $(X_t,B_t)_{t\geq 0}$ be a weak solution of 
\[(\star)=\begin{cases}
    dX_t &= b(t,X_t)dT+\sigma(t,X_t)dB_t\\
    X_0&=x_0\text{ fixed}
\end{cases}\]
Assume there exists $h\in C^{1,2}(\R_+\times\R^d)$ s.t. $h>0$ satisfying 
\begin{equation*}
    (\frac{\partial}{\partial t}+\cL)h=0\forall t,\in [0,T],x\in\R^d
\end{equation*}
where $\cL$ is the generator of the SDE $(\star)$.

By theorem \ref{prop:1.14} $Z_t\coloneqq h(t,X_t)=h(0,X_0)+\int_0^t (\sigma \nabla h)(s,X_s)dB_s$.

$Z_t$ is a positive local martingale. Assume $Z_t$ is a martingale.

W.l.o.g. $Z_0=1$ (if not $h\mapsto \frac{h}{h(0,X_0)}$).

$\implies d\Q_T =Z_T\cdot d\bP$. Let $L_t$ s.t. $Z_t=\exp(L_t-\frac{1}{2}\langle L\rangle_t)$.

Girsanov, since $B_t$ is a $\bP$-BM, 

$\tilde{B}_t\coloneqq B_t-\langle L,B\rangle_t$ is a $Q_T$-BM, where 
\begin{align*}
    \langle L, B\rangle_t =\sigma^\intercal\cdot \nabla\ln h \cdot d B_t
\end{align*}

because of 
\begin{align*}
    dL_t&=\frac{dZ_t}{Z_t}=\frac{\sigma^\intercal\cdot  \nabla h \cdot d B_t}{h(t,X_t)}=\sigma^\intercal\cdot \nabla\ln h \cdot d B_t\\
\end{align*}

And therefore $dB_t = d\tilde{B}_t+\sigma^\intercal \nabla \ln h dt$.

From $(\star)\implies dX_t = (\underbrace{b(t,X_t)+\sigma(t,X_t)\sigma^\intercal(t,X_t)\nabla h(t,X_)}_{\coloneqq \tilde{b}})dt + \sigma(t,X-t)d\tilde{B}_t$

\begin{proposition}\label{prop:2.3}
    Let $X$ be a weak solution of $dX_t=b(t,X_t)+\sigma(t,X_t)dB_t$ on $[0,T]$ under $\bP$.

    Then under $\Q_T, X$ is a weak solution of the SDE 
    \[\begin{cases}
        dX_t &= \tilde{b}(t,X_t)dt+\sigma(t,X_t)d\tilde{B}_t\\
        X_0&=x_0
    \end{cases}\]
    where \dhighlight{$\tilde{b}=b+\sigma\sigma^\intercal \nabla \ln h$}.
\end{proposition}

\begin{remark}
    In applications, often $h(x)\not> 0$ for all $x$!

    E.g.: $h(t,x)=x\implies \left(\frac{\partial}{\partial t}+\cL \right)h=0$ ($X_t=B_t+x_0$), but $h(x)\leq 0$ for $x\leq 0$.

    In this case first do the construction on $[0,\tau]$, where $\tau\coloneqq\inf\{t>0\mid B_t=0\}$

    $\implies$ ok for $B_t^\tau\stackrel{\text{magically}}{\implies}$ the new SDE has drift:\marginnote{We realize there is no problem in the new measure}
    \[\frac{\partial}{\partial x}\ln(h(x))=\frac{1}{h(x)}\frac{\partial h(x)}{\partial x}=\frac{1}{x}\] 
\end{remark}

\begin{aexample} % TODO: Numbering
    If $X_t=B_t$ and $h(t,)=e^{\gamma x-\frac{1}{2}\vert \gamma\vert^2 t}$
    \begin{align*}
        \implies dX_t &= \nabla \ln h(t,X_t)dt + dB_t\\
        &=\nabla(\gamma \cdot x)dt + dB_t \\
        &=\gamma d_t + d B_t
    \end{align*}

    $\implies X_t$ is a BM with drift $\gamma$ w.r.t $\Q$.
\end{aexample}

\section{Diffusion Bridges}

Consider a Markov process\marginnote{This is not true in general, if we don't have unique solutions} $(X_t)_{t\geq 0}$ which is a 
diffusion starting from $X_0=x_0\in\R^d$. % TODO: Was ist eine Diffusion?

We want to condition on the event $\{X_T=y\}$ for some given $T>0,y\in\R^d$.

\dhighlight{Goal:} Find, $\forall y\in \R^d, \Q^y$ which is the conditional measure on $\{X_T=y\}$.

Assume: $(X_t)_{t\geq 0}$ has transition density $P$ s.t. 
\[\forall 0\leq s\leq t\leq T,\bP(X_t\in dz\mid X_s=x)=p(s,x;t,z)dz.\]

$X_t$ solves \[\begin{cases}dX_t &= b(t,X_t)dt+\sigma(t,X_t)dB_t\\
    X_0&=x_0
\end{cases}.\]

Define $h^y(s,x)\coloneqq \frac{p(s,x;T,y)}{p(0,x_0;T,y)}$ for $s\in [0,T),x\in\R^d$.

\begin{remark}
    $s<T$ since otherwise we get ingeneral singularities.
\end{remark}

\begin{lemma}\label{lem:2.4}
    Let $Z_t^y\coloneqq h^y(t,X_t)$ is a martingale. 
\end{lemma}

\begin{proof}
    Let $0\leq <t<T$:
    \begin{align*}
        \bE(Z_t^y\mid\cF_s)&=\bE(h^y(t,X_t)\mid \cF_s)\\
        &\stackrel{\text{MP}}{=}\bE(h^y(t,X_t\mid X_s))\\
        &=\int_{\R^d} h^y(t,x)p(s,X_s;t,x)dx\\
        &=\int_{\R^d} \frac{p(t,x;T,y)}{p(0,x_0;T,y)}p(s,X_s;t,x)dx\\
        &=\frac{1}{p(0,x_0;T,y)\int_{\R^d}p(s,X_s;t,x)p(t,x;T,y)dx}\\
        &\stackrel{\text{Chapman-Kolmogorov}}{=} Z_s^y=
    \end{align*}
    \marginnote{$h^{y}(0,x_0)=1$}
\end{proof}

\beginlecture{07}{02.05.24}

\begin{aremark}
    \dhighlight{Goal:} Find a family $(\Q^y)_{y\in\R^d}$ s.t. for $A\in \cF_T$:
    \[\bP(A)=\bE(\bP(A|X_T))\stackrel{?}{=}\bE(\Q^{X_T}(A))\]
    and $\Q^y\left(\lim_{t\uparrow T} X_t=y\right)=1$
\end{aremark}

\begin{lemma}\label{lem:2.5}
    We can take 
    \[\Q^y(A)=\bE(1_{A}\cdot h^y(s,X_s))\forall A\in\cF_s\]
    for $s\in [0,T]$.
\end{lemma}

\begin{proof}
    For all $A\in\cF_s$, $g$ bounded measurable:
    \begin{align*}
        \bE(1_Ag(X_t))&=\bE(1_A\bE(g(X_T|\cF_s)))\\
        &\stackrel{\text{MP}}{=}\bE(1_A\bE(g(X_T|X_s)))\\
        &=\bE\left(1_A\int_{\R^d}p(s,X_s;T,x)g(x)dx\right)\\
        &=\int_{\R^d}dx g(x)\bE(1_Ap(s,X_s;T,x))
    \end{align*}
    In particular, 
    \begin{align*}
        \bP(A)=\bE(1_A)=\int_{\R^d}dx\bE(1_A\cdot p(s,X_s;T,x))
    \end{align*}
    But 
    \begin{align*}
        \int_{\R^d}dx p(0,X_0;T,x)Q^x(A)&=\int_{R^d} p(0,x_0;T,x)\underbrace{\bE(1_Ah^x(s,X_s))}_{=\int dzp(0,x_0;s,z)h^x(s,z)1_A}\\
        &=\int_{\R^d}\bE(1_Ap(s,X_s;T,x))\\
        &=\bP(A).\qedhere
    \end{align*}
\end{proof}

\begin{remark}
    $Z_t^y=h^y(t,X_t)=h(0,X_0)+\text{ martingale }+\underbrace{\int_0^t( \partial_t +\mathcal{L})h^y(s,X_s)ds}_{=0}.$

    One can verify: $(\partial_t+\mathcal{L})h^y(t,X_t)=0 \iff (\partial_t+\mathcal{L})H^y(t,X_t)=0$ for 
    $H^y(t,x)=p(t,x;T,y)$.

    \begin{align*}
        \frac{\partial}{\partial t}H^y(t,x)&=\lim_{\epsilon\downarrow0}\frac{p(t,x;T,y)-p(t-\epsilon,x;T,y)}{\epsilon}\\
        &=\lim_{\epsilon\downarrow 0} \frac{p(t,x;T,y)-\int \overbrace{p(t-\epsilon,x;t,z)}^{=e^{\epsilon L}(x,z)=\delta_{x-z}+\epsilon\mathcal{L}(x,z)}p(t,z;T,y)dz}{\epsilon}\\
        &=-(\mathcal{L}H^y)(t,x)
    \end{align*}

    $\implies$ this is a doob-h transform with $h=h^y$.

    \begin{align*}
        \frac{d\Q^y}{d\bP}\vert_{\cF_t}=Z_t^y.
    \end{align*}
\end{remark}

\begin{corollary}\label{cor:2.6}
    Assume $(t,x)\mapsto p(t,x;T,y)$ is $C^{1,2}([0,T]\times \R^d)$
    
    $\implies$ The Doob h-transform of the original process under $\Q^y$ satisfies the SDE:
    \begin{align*}
            dX_t=\tilde{b}(t,X_t)dt+\sigma(t,X_t)d\tilde{B}_T
    \end{align*}
    where $\tilde{B}_t$ is a $\Q^y$-BM and $\tilde{b}(t,X_t)=b(t,X_t)+(\sigma\sigma^\intercal \nabla \ln h^y)(t,X_t)$.
\end{corollary}

\begin{remark}
    Take $z\neq y$ and $\forall 0<\epsilon<|z-y|$:
    \begin{align*}
        \Q^y\left(\{|X_t-z|\leq \epsilon\}\right)&=\bE\left(1_{|X_t-z|\leq \epsilon}\frac{p(t,X_t;T,y)}{p(0,x_0;T,y)}\right)\\
        &=\int dx 1_{|x-z|\leq \epsilon}p(0,x_0,t,x)\frac{p(t,x;T,y)}{p(0,x_0;T,y)}\\
        &\stackrel{t\to T}{\longrightarrow}0
    \end{align*}
    $\implies \bP\left(\lim_{t\uparrow T}X_t=y\right)=\bP(X_t=y)=0$, but $\Q^y\left(\lim_{t\uparrow T} X_t=y\right)=1$

    $\implies$ $\Q^y$ is singular w.r.t. $\bP$.
\end{remark}

\begin{aexample}
    $b=\gamma\in\R^d,\sigma=1,X_0=0$ and $T=1$. Under $\bP:$
    \[\begin{cases}
        dX_t=\gamma dt + dB_t\\
        X_0=0
    \end{cases}\]
    Under $\Q^y$?

    \begin{align*}
        h^y(t,x)=\frac{p(t,x;1,y)}{p(0,x_0,1,y)}=\text{fct}(\gamma,y)\cdot \exp\left(-\frac{(y-x)^2}{2(1-t)}\right)\exp\left((y-x)\gamma\right)
    \end{align*}

    $\implies \ln( h^y(t,x))=\ln f(\gamma,y)-\frac{(y-x)^2}{2(1-t)}+(y-x)\gamma$
    $\implies \nabla \ln h^y(t,x)=\frac{y-x}{1-t}-\gamma$.

    Under $\Q^y:$
    \begin{align*}\marginnote{Indepdendent of $\gamma$!}
        dX_t^k&h^y=\left(\gamma_k+\frac{y_k-X_t^k}{1-t}\right)dt + d\tilde{B}_t^k\\
        &=\frac{Y_k-X_t^k}{1-t}dt +d\tilde{B}_t^k
    \end{align*}
    for $k=1,\dots,d$
\end{aexample}

\begin{lemma}\label{lem:2.7}\marginnote{Useful in practice to sample!}
    Let $p$ be the transition density of $X$ w.r.t. $\bP$ and $p^h$ the transition density of $X$ w.r.t. $\Q^y$.

    Let $h^y(t,x)=\frac{p(t,x;T,y)}{p(0,x_0;T,y)}$.

    \begin{align*}\marginnote{Notice how the normalizations inside of $h^y$ cancel!}
        \implies p^h(s,x;t,z)=\frac{1}{h^y(s,x)}p(s,x;t,z)h^y(t,z)
    \end{align*}
\end{lemma}

\begin{proof}
    \begin{align*}
        p^h(s,x;t,z)dz&=\bP(X_t\in dz\mid X_s=x,X_T=y)\\
        &\stackrel{0\leq s\leq t}{=}\frac{p(0,x_0;s,x)p(s,x;t,z)p(t,z;T,y)dz}{p(0,x_0;s,x)p(s,x;T,y)}\\
        &=p(s,x;t,z)\frac{p(t,z;T,y)}{p(0,x_0;T,y)}\frac{1}{\frac{p(s,x;T,y)}{p(0,x_0;T,y)}}dz\qedhere
    \end{align*}
\end{proof}

\section{Brownian motion conditioned to stay positive forever}

\dhighlight{Goal:} $(B_t)_{t\geq 0}$ with $B_0=x_0\geq 0$ and condition on 
\[\{B_t\geq 0\forall t\geq 0\}\]
%TODO Zeichnung
$T_x=\inf\{t\geq 0\mid B_t=x\}$ and take $0<x_0<R$.

$T_0\land T_R=T_R$.

Let $\tilde{T}_R\coloneqq T_0\land T_R$. Define the event $E_R=\{B_{\tilde{T}_R}=R\}=\{T_0\land T_R=T_R\}$.

One verifies $\bP(E_R)=\frac{x_0}{R}\in (0,1)$.

$\bP(E_R)=\bE(1_{B_{\tilde{T}_R}^{x_0}})=\underbrace{\bE(f(B_{\tilde{T}_R}^{x_0}))}_{u(x_0)}$, where 
\[f(x)=\begin{cases}
    1 & x= R\\
    0 & x=0
\end{cases}.\]
$u$ has a link to the PDE:\marginnote{By Theorem 11.6 of the last semester} 
\[\begin{cases}
    \frac{1}{2}\frac{\partial^2}{\partial x^2} u(x)=0&x\in (0,R)\\
    u(x)=f(x) & x\in \{0,R\}
\end{cases}\]

Define the conditional probability \marginnote{later $R\to\infty$}
\[Q^R(A)\coloneqq \frac{\bP(A\cap E_R)}{\bP(E_R)}\]

\begin{lemma}\label{lem:2.8}
    Let $h(x)\coloneqq \frac{\bP_x(E_R)}{\bP_{x_0}(E_R)}=\frac{x}{x_0}$,
    where $\bP_x$ is the unconditional law for $B_0=x$.    
    \[Z_t=h(B_{t\cap \tilde{T}_R})\]
    is a non-negative martingale.
\end{lemma}

\begin{proof}
    \begin{align*}
        \Q^R(A)&\stackrel{A\in\cF_s}{=}\frac{\bE(1_A\bE(1_{E_R}\mid \cF_s))}{\bP(E_R)}=\frac{\bE(1_A\bE(1_{E_R}\cdot 1_{\tilde{T}_R>s}\mid \cF_s))+\bE(1_A\bE(1_{E_R}1_{\tilde{T}_R\leq s}\mid\cF_s))}{\bP(E_R)}\\
        &=\frac{1}{\bP(E_R)}\left[\bE(1_A1_{\tilde{T}_R>s}\underbrace{\bE(1_{E_R}\mid\cF_s)}_{\bP_{B_s}(E_R)})+\bE(1_A\underbrace{1_{E_R}}_{1_{B_{\tilde{T}_R}=R}}1_{\tilde{T}_R\leq s}) \right]\\
        &=\bE(1_A1_{\tilde{T}_R>s}h(B_s))+\bE(1_A1_{\tilde{T}_R\leq s}h(B_{T_R}))\\
        &=\bE(1_A h(B_{s\land \tilde{T}_R}))
    \end{align*}
    $\implies\forall A\in\cF_s: \Q^R(A)=\bE_{\bP}(1_Ah(B_{s\land \tilde{T}_R}))$

    $\implies h(B_{s\land \tilde{T}_R})=\frac{d\Q^R}{d\bP}\vert_{\cF_s} \implies$ $h\in(0,\frac{R}{x_0})$ is a martingale (See construction of Girsanov).  
\end{proof}

\beginlecture{08}{07.05.24}

\beginlecture{09}{14.05.24}

\begin{aexample}
    Discrete time M.C. with transition probability $P$

    aperiodic and irreducible: $\exists n_0\forall n>n_0 (P)_{i,j}^n>0\implies \exists |\lambda_0|\leq 1,|\lambda_1|>|\lambda_2|>|\lambda_3|$:\[(P)^n\underbrace{\varphi_0(i)}_{>0}=\lambda_0\varphi_0(i)\]

    $\lim_{n\to\infty} \underbrace{P}_{(1+L)}^n = (\varphi_0(1),\dots,\varphi_0(d))$

    where $(1+L)\to e^{tL}$, for which the real eigenvalues are positive.

\end{aexample}


\section{Diffusion conditioned to stay in a domain} % should be 2.6

Domain $D\subset\R^d$: bounded, open, connected.

Diffusion with generator 
\[L=\frac{1}{2}\sum_{i,j=1}^d a_{ij}\frac{\partial^2}{\partial x_i\partial x_j}+\sum_{i=1}^d b_i \frac{\partial}{\partial x_i}\] 
$\iff$ SDE 
\[\begin{cases}
    dX_t=b(t,X_t)dt+\frac{1}{2}\sigma(t,X_t)dB_t\\
    X_0=x_0
\end{cases}\]
$a=\sigma\sigma^\intercal$.

Assume: $b,\sigma$ are continuous, $\sigma\in C^1$ and let $\tau_D=\int\{t\geq 0|X_t\notin D\}$

\dhighlight{Key assumption} in Pinsky's paper:
\begin{enumerate}
    \item[(a)] $\bP_x(\tau_D>t)\in\C^2(D)$
    \item[(b)] \[\begin{cases}
        \bP_x(\tau_D>t)&=C_1\varphi_0(x)e^{-\lambda_0 t}+ o\left(e^{-\lambda_0t}\right)\\
        \nabla \bP_x(\tau_d>t)=C_1\nabla \varphi_0(x)e^{-\lambda_0t}+o\left(e^{-\lambda_0t}\right)
    \end{cases}\] where \[\begin{cases}
        -L\varphi_0(x)=\lambda_0\varphi_0(x) & x\in D\\
        \varphi_0(x)=0 & x\in\partial D
    \end{cases}\] and $\lambda_0$ is the smallest eigenvalue of $-L$. \marginnote{in the non symmetric case take the real part first}
    Which means $\bP(\tau_d>t)\stackrel{t\to \infty}{\to} 0$
\end{enumerate}

Want: Condition $X$ to stay in $D$ forever.

\begin{enumerate}
    \item[(1)] Condition $X_t$ to $\{X_t\in D:0\leq t\leq T\}$: Forall $A\in\cF_T$: define a measure \[\Q^T(A)\coloneqq \frac{\bP_{x_0}(A\cap \{\tau_D>T\})}{\bP_{x_0}(1_{\tau_D}>T)}=\frac{\bE_{x_0}(1_A\cdot 1_{\tau> T})}{\bE_{x_0}(1_{\tau>T})}\] 
\end{enumerate}

\begin{lemma}\label{lem:2.10}
    $\forall s<T,A\in\cF_s$,
    \[\Q^T(A)=\bE_{x_0}(1_A\cdot Z_s^T)\]
    where $Z_s^T=\frac{g^{T-s}(X_{s\land \tau_D})}{g^T(x_0)}$ with $g^t(x)\coloneqq \bP_x(\tau_D>t)$.
\end{lemma}

\begin{proof}
    Let $A\in\cF_s,s<T$:

    \begin{align*}
        \Q^T(A)&=\frac{\bE_{x_0}(1_A1_{\tau_D>T})}{g^T(x_0)}=\frac{\bE_{x_0}(1_A\bE_{x_0}(1_{\tau_D>T}|\cF_s))}{g^T(x_0)}\\
        &\stackrel{\text{M.P.}}{=}\frac{\bE_{x_0}(1_A\bE_{x_0}(1_{\tau_D>T}|X_s))}{g^T(x_0)}\\
        &=\frac{1}{g^T(x_0)}\left[\bE_{x_0}(1_A\underbrace{\bE_{x_0}(1_{\tau_D>T}1_{\tau_D>s}|X_s)}_{g^{T-s}(X_s)})+\bE_{x_0}(1_A\underbrace{\bE_{x_0}(1_{\tau_D>T}1_{\tau_D<s}|X_s)}_{=0})\right]\\
        &=\frac{1}{g^T(x_0)}\left(1_A g^{T-s}(X_{s\land \tau_D})\right)
    \end{align*}
\end{proof}

\begin{lemma}\label{lem:2.11}
    $Z_0^T=1$ and $(Z_s^T)_{s\in [0,T]}$ is a martingale.
\end{lemma}

\begin{proof}
    By lemma \ref{lem:2.10} $\implies Z_s^T=\frac{d\Q^T}{d\bP}\mid_{\cF_s}\to$ is a martingale.
\end{proof}

\begin{remark}
    By construction, $\Q^T(\tau_D\leq T)=\frac{1}{g^T}(x_0)\bE_{x_0}(1_{\tau_D\leq T}1_{\tau_D>t})=0$
\end{remark}

Assume that $g^t(x)$ is $C^{1,2}$ (1 in time, 2 in space) $\implies$ apply Itô and 
doob transform gives:

\begin{proposition}\label{prop:2.12}
    Let $X$ be a weak solution of $dX_t=b(t,X_t)dt+\sigma(t,X_t)dB_t$ under $\bP$.

    $\implies$ $X$ is a weak solution of 
    \[dX_t=\left(b(t,X_t)+\frac{a(t,X_t) \nabla g^{T-t}(X_t)}{g^{T-t}(X_t)}\right)dt +\sigma(t,X_t)d\tilde{B}_t, 0\leq t\leq T\]
    under $\Q^T$ provided $g^t(x)>0\forall x\in D,t\geq 0$.
\end{proposition}

What happens in the $T\to\infty$ limit?

By assumption (b)\marginnote{Their operator is not the same as ours. The only difference is a drift term.}

$\lim_{T\to\infty}\frac{\nabla g^{T-t}(x)}{g^{T-t}(x)}=\frac{\nabla \varphi_0(x)}{\varphi_0(x)}$

\begin{remark}
    If $g\in C^{1,2}$, it satisfies the parabolic PDE 
    \[(\star)\begin{cases}\frac{\partial}{\partial t}g^t(x)=L g^t(x) & x\in D\\ g^t(x)=0 & x\in\partial D\end{cases}\]
    Apply Theorem 11.5 from the WS with $A=L,u(t,x=g^t(x))$: 
    $u(0,x)=\bE_x(1_{\tau_D>0})=0,u(t,x)=\bE_x(1_{\tau_D>t})=0\forall t,x\in\partial D$.
\end{remark}

Pinsky proved that $\lim_{T\to\infty} Q^T=Q$ weak and that under $Q$ the process satisfies
\[dX_t = \left[b(t,X_t)+a(t,X_t)\frac{\nabla \varphi_0(X_t)}{\varphi_0(X_t)} \right]dt +\sigma(t,X_t)dW_t.\]


Why is $\bP_x(\tau_D>t)=C_1\varphi_0(x)e^{-\lambda_0 t}+o(e^{-\lambda_0t})$

\begin{aremark}

    This is not the case for BM conditioned to stay in $D=(0,\infty)$. Reason: Spectrum of $-L$ is $\R_+$

\end{aremark}

Since it satisfies the PDE $(\star)$, $g^t(x)=(e^{tL})g^0(x)$.

if the spectrum of $-L$ is discrete with eigenvalues $0\leq \lambda_0<\lambda_1\leq \dots$, where 
there is a spectral gap between $\lambda_0$ and $\lambda_1$: $\exists$ eigenfunctions $\varphi_0(x),\varphi_1(x),\dots$,
normalized  as $\Vert \varphi_n\Vert_{L^2(D)}=1\implies +L\varphi_n(x)=-\lambda_n\varphi_n(x)$
$\implies$ we can choose $\varphi_0,\varphi_1,\dots$ to be orthonormal: $(\varphi_i,\varphi_j)_{L^2(D)}=\delta_{ij}$. 

$\implies 1=\sum_{n\geq 0}\varphi_n\varphi_n^*$, since $\varphi_n\varphi_n^*$ is the projection onto 
the space generated by $\varphi_m$:
\[\varphi_n\varphi_n^* f=\varphi_n (\varphi_n,f)_{L^2(D)}\]

\begin{align*}
    f(L)\varphi_n&=f(-\lambda_n)\varphi_n\\
    g^t(x)&=(e^{tL})g^0(x)=e^{tL}1g^0(x)=\sum_{n\geq 0}e^{tL}\varphi_n(x)\varphi_n^*g^0=\sum_{n\geq 0} e^{-\lambda n t}\varphi_n(x)(\varphi_m,g^0)\\
    &=(\varphi_0,g^0)\varphi_0(x)e^{\lambda_0t}+\underbrace{e^{-\lambda_0 t}\sum_{n\geq 0}e^{(\lambda_0-\lambda_n)t}\varphi_n(x)(\varphi_n,g^0)}_{o(e^{-\lambda_0 t})}
\end{align*}


\begin{example}
    One dimension.

    $L=\frac{1}{2}\frac{d^2}{dx^2}$ and $D=[0,R]$.

    Solve $-L\varphi(n)(x)=\lambda_n\varphi_n(x)$ with $\varphi_n(x)=0$ for $x\in \{0,R\}$.

    Solutions $\varphi_n(x)=C_1\sin\left(\frac{\pi x}{R}\cdot (n+1)\right)$
    and 
    $\lambda_n=\left(\frac{\pi(n+1)}{R}\right)^2\implies \lambda_0=\frac{\pi^2}{R^2}$.

    For $x\in(0,R):$\begin{align*}
        \bP_x(\tau>t)\approx \tilde{C}_1 \varphi_0(x)e^{-\frac{\pi^2 t}{R^2}}
    \end{align*}

    \[\implies dX_t=\frac{\pi}{R}\frac{\cos(\frac{\pi X_t}{R})}{\sin(\frac{\pi X_t}{R})}d+dW_t\]

    as $X_t\to 0$ (or $X_t\to\R$), drift $\approx \frac{1}{x_t}$ or $(\frac{-1}{R-X_t})$.

    Also: formally $R\to\infty$ in the SDE, we get $dX_t=\frac{dt}{X_t}+dW_t$, which is the SDE 
    we derived for BM conditioned to stay $>0$ forever.
\end{example}


\beginlecture{10}{16.05.24}

\begin{example}[Brownian Motion in a Weyl chamber]\label{ex:2.14}
    The \dhighlight{Weyl chamber} is defined as \[W^d=\{x\in\R^d\mid x_1<x_2<\dots<x_d\}.\]
    Given a $d$ dim. Brownian motion $(B_t)_{t\geq 0}$ with $B_0\in W^d$ what is the sde of 
    this BM conditioned on staying in $W^d$ forever.

    What is the harmonic function vanishing at $\partial W^d$?

\end{example}

\begin{lemma}\label{lem:2.15}
    $h(x)\coloneqq \prod_{1\leq k <l\leq d}(x_l-x_k)$ is harmonic and satisfies 
    \[\frac{1}{2}\sum_{k=1}^d\frac{\partial^2}{\partial x_k^2}h(x)=0,x\in W^d\]
    and $h(x)=0$ for $x\in \partial W^d$, $h(x)>0$ for $x\in W^d$.
\end{lemma}

\begin{remark}
    $h(x)=\det(x_k^{l-1})$
\end{remark}

\begin{proof}
    Last two properties are clear. $h$ is a polynomial, antisymmetric in each pair $x_l-x_k$ and has 
    lowest possible power.

    $\Delta h(x)$ is still antisymmetric, but with lower power $\implies$ $\Delta h(x)=0$.
\end{proof}

The BM conditioned to stay in the Weyl chamber will satisfy the SDE:
\begin{align*}
    dX_t^k &= \sum_{l\neq k}\frac{dt}{X_t^l-X_t^k}+dB_t^k, & 1\leq k\leq d
\end{align*}

\section{Stationary distribution for diffusions}

let $dX_t=b(X_t)dt+\sigma(X_t)dB_t$, where $b,\sigma$ are time independent with generator 
\[L=\sum_{k=1}^db_k(x)\frac{\partial}{\partial x_k}+\frac{1}{2}\sum_{k,l=1}^da_{k,l}(x)\frac{\partial^2}{\partial x_k\partial x_l}\]


\begin{definition}\label{def:2.16}
    A probability measure $\mu$ \dhighlight{stationary (or invariant)} if for all $f\in C_0^\infty(\R^d)$\marginnote{or only test schwarzfunctions}
    \[\int_{\R^d} (Lf)(x)\mu(dx)=0\]
\end{definition}

Assume that $\mu\ll$ Lebesgue, i.e., $\mu(dx)=\rho(x)dx$ for some positive function $\rho(x)\in C^2$ with $\int\rho(x)dx=1$.

\begin{lemma}\label{lem:2.16b}
    $\mu$ is stationary with density $\rho$\marginnote{this also assumes $b$ to be differentiable!}
    \[\iff L^*\rho(x)=0\text{ almost everywhere}\]
    where $L^*$ is the adjoint of $L$, given by 
    \[L^* \rho(x)=\frac{1}{2}\sum_{k,l=1}^d\frac{\partial^2}{\partial x_l \partial x_l}(a_{k,l(x)\rho(x)})-\sum_{k=1}^d\frac{\partial}{\partial x_k}(b_k(x)\rho(x))\]
\end{lemma}

\begin{proof}
    \begin{align*}
        &\int_{\R^d}(L f)(x)\rho(x)dx=0\\
        =&\int_{\R^d}dx_1,\dots,dx_d\rho(x)\left(\frac{1}{2}\sum_{k,l=1}^d\frac{\partial^2}{\partial x_k\partial x_l} f(x)+\sum_{k=1}^d b_k(x)\frac{\partial}{\partial x_k}f(x)\right)
    \end{align*} 
    \begin{align*}
        \int_{R^{d-1}}\prod_{l\neq k} dx_l\underbrace{\int_R dx_k \rho(x)b_k(x)\frac{\partial }{\partial x_k}f(x)}_{\stackrel{\text{I.b.p}}{=}-\int_R dx_k\frac{\partial}{\partial x_k}(\rho(x)b_k(x)) \cdot f(x)}
    \end{align*}   
\end{proof}

\begin{example}[1-dim. diffusion]\label{lem:2.17}
    Let $L=b(x)\frac{\partial }{\partial x}+\frac{1}{2}\sigma^2(x)\frac{\partial}{\partial x^2}$ for all $x\in\ R$.
    \[\implies L^*\rho(x)=\frac{1}{2}\frac{\partial^2}{\partial x^2}(\rho(x)\sigma^2(x))-\frac{\partial}{\partial x}(\rho(x)b(x))\stackrel{\text{wanted}}{=}0\]
    $\implies \exists c_1$ s.t. $\frac{1}{2}\frac{\partial}{\partial x}(\underbrace{\rho(x)\sigma^2(x)}_{=g(x)})-\underbrace{\rho(x)b(x)}_{=g(x)\frac{b(x)}{\sigma^2}}=c_1$.

    Let $s(x)=\int_{x_0}^xdye^{-\int_{x_0}^y dz\frac{2b(z)}{\sigma^2(z)}}$ be the \dhighlight{scale function}.

    The equation for $c_1$ is equivalent to 
    \begin{align*}
        s'(x)\sigma^2(x)\rho(x)=c_2+2c_1s(x)
    \end{align*}
    for some constant $c_2$,$s'(x)e^{-\int_{x_0}^y dz\frac{2b(z)}{\sigma^2(z)}}=>0$.

    s satisfies $\frac{1}{2}\sigma^2(x)s''(x)+b(x)s'(x)=0$, for the Bessel process $d=2$, $s(x)=\ln(x)$

    If $s(\R)=\R$, then $c_1=0$, which implies 
    \[\rho(x)=\frac{c_2}{\sigma^2(x)}\underbrace{e^{\int_{x_0}^x\frac{2b(y)}{\sigma^2(y)}dy}}_{\frac{1}{s'(x)}}\]
    which satisfies positivity.
\end{example}

\begin{lemma}\label{lem:2.18}
    If $S(\R)=\R$, then there exists stationary measure with density $\rho(x)$ as described above.
\end{lemma}

\dhighlight{Counterexample:}
\begin{itemize}
    \item $\sigma=1$, $x_0=0$, $b(z)=b>0$.
\end{itemize} 

$\implies s(x)=\frac{1-e^{-2bx}}{2b}\to s'(x)=e^{-2bx}$.

\begin{itemize}
    \item $s(-\infty)=-\infty$
    \item $s(\infty)=\frac{1}{2b}$
\end{itemize}

$\implies $ Argument for $c_1=0$ does not work.

\[\implies \rho(x)=\tilde{c}_1+\tilde(c)_2e^{2bx}\]
which can't be a density, because if at least $\tilde{c}_1$ or $\tilde{c}_2$ are $\neq 0$
$\implies$ it is not integrable.


\begin{example}\label{ex:2.19}
    Let $Lf(x)=\frac{1}{2}\frac{\partial^2}{\partial x^2} f(x)+b(x)\frac{\partial }{\partial x}f(x)$
    where $b(x)=\frac{\partial}{\partial x}\ln h(x)$ for some $h(x)>0$.


    Assume $h$ is normalized as $\int_R (h(x))^2 dx=1$.\marginnote{Implicitly assumes $h\in L^2(\R)$} 
    $\implies$ \dhighlight{Claim:} The stationary density of the process with generator $L$ is given by 
    $\rho(x)=(h(x))^2$.

    \underline{Verify the claim:}

    \begin{align*}
        L^*\rho&(x)\stackrel{?}{=}0\\
        &=\frac{1}{2}\frac{\partial^2}{\partial x^2} \rho(x)-\frac{\partial}{\partial x}(b(x)\rho(x))\\
        &=\frac{1}{2}\frac{\partial^2}{\partial x^2} (h(x))^2-\frac{\partial}{\partial x}(\frac{h'(x)}{h(x)}h(x)^2)\\
        &=\frac{\partial}{\partial x} (h(x)h'(x))-\frac{\partial}{\partial x}(h'(x)h(x))=0
    \end{align*}

    Example \ref{ex:2.13} $\implies h(x)=c\sin(\frac{\pi x}{L})$\marginnote{here $L\in\R$ is not the operator!}

    \begin{align*}
        \int_0^L h(x)^2dx=c^2\int_0^2\sin^2(\frac{\pi x}{L})dx=1\\
        \implies c=\sqrt{\frac{2}{L}}\implies \rho(x)=\frac{2}{L}\sin^2(\frac{\pi x}{L}).
    \end{align*}
\end{example}


\section{Uniqueness in law and path integral formula}

Consider the SDE \[\begin{cases}
    dX_t=b(t,X_t)dt+dB_t & \text{in }\R^d\\
    x_0=x_0
\end{cases}\]

Assume \begin{equation*}\
   (\star) \forall T>0 \int_0^T|b(s,X_s)|^2ds < \infty\as
\end{equation*}.

\dhighlight{Goal:} Show uniqueness in law.

Consider any $(X,\mathcal{B},\bP)$ weak solution satisfying $(\star)$ 
and define\[\tau_n\coloneqq \inf\{t\geq 0\mid \int_0^t|b(s,X_s)|^2ds\geq n\}\]
which, by $(\star)$ goes to infinity.

For each $n$, define $\Q^n:$

\[\frac{d\Q^n}{d\bP}=e^{-\underbrace{\int_0^{\tau_n}b(s,X_s)dB_s}_{L_{\tau_n}}-\frac{1}{2}\int_0^{\tau_n}|b(s,X_s)|^2ds}.\]

By Girsanov: $\tilde{B}_t\coloneqq B_t-\langle B,L\rangle_t=B_t+\int_0^{t\land \tau_n} b(s,X_s)ds$
is BM w.r.t. $\Q^n$, which implies 

$X_t=\tilde{B}_t$ for all $t\leq \tau_n$ w.r.t. $\Q^n$.

Let events of $(X,B)$ $\cF_T$ measurable for some time $T$: are in $A_T$.

\begin{align*}
    \bE_\bP\left(1_{(X,B)\in A_T}1_{T\leq \tau_n}\right)&=\bE_{\Q^n}\left(1_{(X,\mathcal{B})\in A_T}1_{T\leq \tau_n}e^{\int_0^{\tau_n} b(s,X_s)\underbrace{dB_s}_{=dX_s-b(s,X_s)ds}+\frac{1}{2}\int_0^{\tau_n}|b(s,X_s)|^2ds}\right) \\
    &=\bE_{\Q^n}\left(1_{(X,B)\in A_T}1_{T\leq \tau_n}e^{\int_0^{\tau_n} b(s,X_s)d X_s-\frac{1}{2}\int_0^{\tau_n}|b(s,X_s)|^2ds}\right) 
\end{align*}

$B_s$ is adapted to $X_s$ $\implies$ for some $\Phi$: $B=\Phi(X)\implies$

\begin{align*}
    \bE_{\Q^n}\left(1_{(X,\Phi(X))\in A_T}1_{T\leq \tau_n}e^{\int_0^{\tau_n} b(s,X_s)d X_s-\frac{1}{2}\int_0^{\tau_n}|b(s,X_s)|^2ds}\right)    
\end{align*}

But: Law of $X$ w.r.t. $\Q^n$ is the law of $BM$. Take $n\to\infty$, then $\Q^n\to$ Wiener measure,
$T\to\infty \implies$ we look at every possible event, but the probability does only depend on $X$!

\[\mathbb{P}((X,B)\in\mathcal{B}(\mathcal{E}^d\times\mathcal{E}^d))\]





