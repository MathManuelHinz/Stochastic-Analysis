\chapter{Weak solutions of SDEs}

\underline{SDE:} 
\begin{equation}\label{eq:001}
    \begin{cases}
        dX_t=b(t,X_t)dt+\sigma(t,X_t)dB_t\\
        X_0=x_0 \highlight{ Initial condition}
    \end{cases}
\end{equation}
\begin{itemize}
    \item $X_t\in\R^d,B$ : $n$-dim BM 
    \item $b(t,x): [b_k(t,x)]_{1\leq k\leq d}$: \highlight{drift vector}
    \item $\sigma(t,x)=[\sigma_{k,l}(t,x)]_{\stackrel{1\leq k\leq d}{1\leq l\leq n}}$: \highlight{dispersion matrix}
    \item $a(t,x)=\sigma(t,x)\cdot\sigma(t,x)^\intercal$: \highlight{diffusion matrix}
\end{itemize}

\section{Strong solutions}

\begin{definition}\label{def:strong_solution}
    Let $(\Omega,\cF,(\cF)_{t\geq 0},\mathbb{P})$ be a filtered probability space
    with $\cF_t=\sigma(x_0,(B_s)_{0\leq s\leq t})$
    \begin{itemize}
        \item a $d$-dim process $X_t$ is a \highlight{strong solution} of equation \ref{eq:001} if: \begin{itemize}
            \item $X_t=x_0$ a.s. 
            \item $X_t$ is adapted to $\cF_t\forall t\geq 0$
            \item $X$ is a continuous semimartingale s.t. $\forall t\geq 0$:\[\int_0^t(\Vert b(s,X_t)\Vert+\Vert \sigma(t,X_t)\vert^2)ds <\infty \as\]
            \item $X_t=x_0+\int_0^t b(s,X_s)ds+\int_0^t\sigma(s,X_s)dB_s$
        \end{itemize}
    \end{itemize}
\end{definition}

In the last semester we proved:
\begin{theorem}\label{thm:existance_strong_solution}
    Assume that $b,\sigma$ are globally lipschitz \marginnote{Foundations of Stochastic Analysis Thm 8.6}
    with at most linear growth at $\infty$ (in space)

    $\implies \exists!$ strong solution of SDE.
\end{theorem}

\begin{aremark}
    There exists $K>0$ s.t. for all $x,y\in\R^d$:
    Globally Lipschitz: 
    \[\Vert b(t,x)-b(t,y)\Vert+\Vert \sigma(t,x)-\sigma(t,y)\Vert\leq K\Vert x-y\Vert\]
    Linear growth condition:
    \[\Vert b(t,x)\Vert+\Vert \sigma(t,x)\Vert\leq K(1+\Vert x\Vert)\]
\end{aremark}

\begin{remark}
    For strong solutions, $\cF_t$ is given by the driving BM, wich is given to us.

    $\implies X_t(\omega)=\Phi(x_0(\omega),(B_s)_{0\leq s\leq t})$ 
\end{remark}

\section{Weak solutions}

\begin{itemize}
    \item For weak solutions we do not fix the driving brownian motion.
\end{itemize}

\begin{definition}\label{def:weak_solution}
    A \dhighlight{weak solution} of equation \ref{eq:001} is a \highlight{pair} of adapted processes 
    $(X,B)$ to a $(\Omega,\cF,(\cF_t)_{t\geq 0},\mathbb{P})$ s.t.\begin{itemize}
        \item $B$ is a $n$-dim BM 
        \item $X$ is a $d$-dim continuous semimartingale with \begin{enumerate}
            \item $X_0=x_0\as$
            \item $\forall t\geq 0$\[\int_0^t(\Vert b(s,X_t)\Vert+\Vert \sigma(t,X_t)\vert^2)ds <\infty \as\]
            \item $X_t=x_0+\int_0^t b(s,X_s)ds+\int_0^t\sigma(s,X_s)dB_s$
        \end{enumerate}
    \end{itemize}
\end{definition}

\begin{remark}
    \begin{itemize}
        \item The filtration $(\cF_t)_{t\geq 0}$ is not necessarily the one generated by $B$
        \item If $X$ is adapted to the filtration generated by the BM $\implies$ we have strong solutions
        \item $\exists$ weak solution which are not strong solutions
        \item Warning: To give a weak solution we really need to provide  $(\Omega,\cF,(\cF_t)_{t\geq 0},\mathbb{P},X,B)$ 
    \end{itemize}
\end{remark}

\begin{definition}[Uniqueness in law]\label{def:uniqueness_in_law}
    An SDE \ref{eq:001} has \dhighlight{uniqueness in law} if given any two weak solutions $(\Omega,\cF,(\cF_t)_{t\geq 0},\mathbb{P},X,B)$
    and $(\tilde{\Omega},\tilde{\cF},(\tilde{\cF}_t)_{t\geq 0},\tilde{\mathbb{P}},\tilde{X},\tilde{B})$ satisfy:\marginnote{They agree on any set in the sigma algebra}
    \[\text{Law}_{\mathbb{P}}(X)=\text{Law}_{\tilde{\mathbb{P}}}(\tilde{X})\]
\end{definition}

\begin{definition}[Pathwise uniqueness]\label{def:pathwise_uniqueness}
    The SDE \ref{eq:001} has pathwise uniqueness if, whenever
    the filtered space $(\Omega,\cF,(\cF_t)_{t\geq 0},\mathbb{P})$
    and  $(B_t)_{t\geq 0}$ are fixed, then two solutions $X,\tilde{X}$
    with $X_0=\tilde{X}_0$ are indistinguishable.
\end{definition}

\begin{example}[No strong solutions, no pathwise uniqueness, $\exists$ weak solution \& and uniqueness in law by Tanaka]
    \begin{equation}
        \begin{cases}
            dX_t=\text{sgn}(X_t)dB_t\\
            X_0=0
        \end{cases}
    \end{equation} or more generally $X_0=Y$, where 
    \begin{equation*}
        \text{sgn}(x)=\begin{cases}
            1& x>0\\-1 &x\leq 0
        \end{cases}
    \end{equation*}
    Let $W$ be a BM with $W_0=Y$. Define 
    \[B_t\coloneqq \int_0^t\text{sgn}(W_s)dW_s\text{ or } dB_t=\text{sgn}(W_t)dW_t\]
    \begin{align*}
        \implies dW_t &= \text{sgn}(W_t)dB_t \\
        \implies W_t&=y+\int_0^t \text{sgn}(W_s)dB_s
    \end{align*}
    $B_t$ is a local martingale with 
    \[\langle B\rangle_t=\int_0^t \underbrace{(\text{sgn}(W_s))^2}_{=1} \underbrace{d\langle W\rangle_s}_{=ds}=t\]
    Also $B_0=0$, therefore $B$ is a BM (see Lévy characterization) $\implies W$ solves the SDE.
    For $Y=0$, $W$ \highlight{and} $-W$ solves the same SDE.

    $\implies$
    \begin{itemize}
        \item  exists weak solutions
        \item For $Y=0$: no pathwise uniqueness
        \item Uniqueness in law (because the law is determined by $X_t$ being a BM)
        \item No strong solution, because: $\cF^B=\cF^{|W|}\subsetneqq \cF^{W}$\marginnote{Not a proof just yet, just the reason!} 
    \end{itemize}
\end{example}

\begin{example}[No solutions]\label{ex:1.8}
    \begin{equation}
        \begin{cases}
            dX_t=-\frac{1}{2X_t}1_{X_t\neq 0}dt+dB_t\\
            X_0=0
        \end{cases}        
    \end{equation}
    Assume there exists a solution. Use Itô formula for $X_t^2$, then:
    \begin{align*}
        X_t^2&=2\int_0^t X_s dX_s +\int_0^t 1 ds\\
        &=-\int_0^t 1_{X_s\neq 0}ds +2\int_0^t x_s dB_s+t\\
        &=\int_0^t 1_{X_s=0}ds+2\int_0^t X_s dB_s
    \end{align*}
    We will prove $\int_0^t 1_{X_s=0}ds=0$ $\implies X_t^2$ is a local martingale, $X_t^2\geq 0$ (and therefore a supermartingale) and $X_0=0$ ($\implies \mathbb{E}(X_t^2)=0$). 
    If $X_t=0\implies \int_0^t 1_{X_s=0}ds=t\implies 0=dB_t$ which are contradictions!
\end{example}

\begin{remark}
    If $X=\underbrace{M}_{\in\Mloc}+\underbrace{A}_{\in\cA}$
    \[\implies \forall b\in \R\int_0^t 1_{X_s=b}d \langle M\rangle_s=0\]
\end{remark}

{\huge\dhighlight{Sheet 0:}}
\begin{itemize}
    \item Mail: Alexander.Becker@uni-bonn.de
    \item Exercise sheet handed in Fr 10 am via eCampus
    \item Groups of 3
\end{itemize}

Motivation:

in the last semester: Introduction to stochastic analysis
\begin{itemize}
    \item Diffusion processes \& SDEs
\end{itemize}
Here: Deepen the knowledge \& have fun
\begin{itemize}
    \item Learn SDE techniques and get other results
    \item Modify diffusion processes to behave in a certain way
    \item Ex. diffusion bridge (Brownian bridge)
    \item Condition a diffusion to stay in a domain\begin{itemize}
        \item Ex. Condition BM to stay positive
        \item Old SDE: $dB_t=dB_t$
        \item New SDE: $dX_t = \frac{1}{X_t}dx+dB_t\to P(X_t\in\cdot)=P(B_t\in\cdot|B>0\text{ forever})$
        \item $D\subset\R^d$ open domain, $X$ diffusion process, with generator $L=\Sigma b\partial_i+\frac{1}{2}\sum a^{ij}\partial_i\partial_j$ \[Y\coloneq( X|X\in D\text{ forever})\] get drift term $\nabla \log\phi_0$, where $\phi_0$ is the lowest eigenfuncion of $-L$ on $D$ with dirichlet boundary.
    \end{itemize}
\end{itemize}

\dhighlight{Recap:}

Brownian motion:

\begin{adefinition}
    $B_0=0$, independent \& $\mathcal{N}(0,t_i-t_{i-1})$ increments, $t\mapsto B_t(\omega)$ continuous.
\end{adefinition}

Regularity of path $t\mapsto B_t(\omega)$:
\begin{itemize}
    \item nowhere differentiable 
    \item $\alpha-$locally Hölder continuous $\iff \alpha<\frac{1}{2}$
    \item Quadratic variation $\langle B\rangle_t=t$
    \item Generator $\frac{\Delta}{2}$
    \item Recurrent $\iff ds2$?
\end{itemize}

Itô-Integral:

\begin{enumerate}
    \item If $X$ simple process $\implies$ RS-Integral
    \item Itô isometry $\mathcal{E}\to\{L^2-\text{local martingale}\},\cE\subset L^2(d[M])$ dense
    \item general $X:$ $\int XdM$ as $L^2$-limit
\end{enumerate}
\begin{aremark}[Itô formula]
    \begin{itemize}
        \item \[df(t,X_t)=\partial_t f(t,X_t)dt+\partial_xf(t,X_t)dX_t+\frac{1}{2}f(t,X_t)d\langle X\rangle_t\]
        \item associative $\int Xd(\int Y dZ)=\int XYdZ$
        \item If $M$ local martingale $\implies \int XdM$ local martingale
    \end{itemize}
\end{aremark}

SDEs:

$DX_t=b(t,X_t)dt+\sigma(t,X_t)dB_t$

\begin{itemize}
    \item ex./ uniqueness: $b,\sigma$ locally Lipschitz $\implies$ strong ex.+pathwise uniqueness until explosion time
    \item globally Lipschitz (in space) and linear growth ($|b(t,x)|+|\sigma(t,x)|\leq C(1+|x|)$), $e=\infty$.
\end{itemize}

\includeproblem{00.1: SDE}{problems/00_1}
\includesolution{00.1}{solutions/00_1}
\includeproblem{00.2: Time change}{problems/00_2}
\includesolution{00.2}{solutions/00_2}
\includeproblem{00.3: SDE and PDE}{problems/00_3}
\includesolution{00.3}{solutions/00_3}



\beginlecture{02}{16.04.24}

\begin{example}[Non-uniqueness of law, non pathwise uniqueness, with solutions]
    \[\begin{cases}
        dX_t &= 1_{X_t\neq 0}dB_t\\
        x_0&=0
    \end{cases}\]
    Then \[X_t=0\forall t\geq 0\] and
    \[X_t=B_t\forall t\geq 0\] 
    both are solutions:\marginnote{$X_t=\int_0^t 1_{X_s\neq 0}dB_s$\\
                                   $=\int_0^t (1-1_{X_s=0}dB_s)$\\
                                   $B_t-\int_0^t1_{X_s=0}dB_s$}
    \begin{align*}
        X_t-B_t=-\int_{0}^t1_{X_s=0}dB_s\implies \langle X-B\rangle_t =\int_0^t 1_{X_s=0}d\langle B\rangle_s=0
    \end{align*}

    Let $\eta\sim\text{Ber}\left(\frac{1}{2}\right)$ independent of $(B_t)_{t\geq 0}$
    and define 
    \[\tilde{X}_t=\begin{cases}
        0 & \eta=0\\
        B_t & \eta=1
    \end{cases}\]

    $\implies \tilde{X}_t$ is adapted to $\sigma(\eta(B_s)_{0\leq s\leq t})$, but not to $\sigma((B_s)_{0\leq s\leq t})$ and therefore not a 
    \hyperref[def:strong_solution]{strong solution}.

\end{example}

\begin{example}[No strong solution, weak solution, no pathwise uniqueness, no uniqueness in law]\label{ex:1.9}
    \[\begin{cases}
        dX_t&=1_{X_t\neq 1}\sgn(X_t)dB_t\\
        X_0&=0
    \end{cases}\]
    Let $Y_t$ be a solution of \[\begin{cases}
        dY_t&=\sgn(Y_t)dB_t\\
        Y_0&=0
    \end{cases}\]
    $\implies X_t\coloneqq Y_{t\land \tau}$, where $\tau=\inf\{s\geq 0\mid Y_s=1\}$ is also a solution.
\end{example}

\begin{theorem}[Yamada-Watanabe]\label{thm:1.10-Yamada-Watanabe}
    If both existence of weak solutions and pathwise uniqueness hold, uniqueness in law also holds.
    
    Moreover, $\forall$ choices of $(\Omega,\cF,(\cF_t)_{t\geq 0},\bP)$ and $(B_t)_{t\geq 0}$ then there exists a \hyperref[def:strong_solution]{strong solution}.
\end{theorem}

\section{Lévy characterization}

\begin{example}\label{ex:1.11}
    Consider the SDE 
    \[\begin{cases}
        dX_t&=O_tdB_t\\
        X_0=x_0
    \end{cases}\]
    where both $X_t$ and $B_t$ are $d$-dimensional and \marginnote{i.e. $O_t$ is a rotation}
    $O_t$ is an adapted process (matrix) s.t. $O_t^\intercal O_t=1\forall t\geq 0$

    \begin{align*}
        \implies X_t^k& = X_0^k+\sum_{l=1}^d\int_0^t O_s^{k,l}dB_s^l\\
        \implies \langle X^k,X^{\tilde{k}}\rangle_t&=\sum_{l,\tilde{l}}^d\int_0^tO_s^{k,l}O_s^{\tilde{k},\tilde{l}}d\langle B^l,B^{\tilde{l}}\rangle_s=\sum_{l=1}^d\int_0^t O_s^{k,l}O_s^{\tilde{k},l}ds \\
        &=\int_0^t \underbrace{\left(O_sO_s^\intercal\right)^{k,\tilde{k}}}_{=1}ds=\delta_{k,\tilde{k}}t\stackrel{\text{Lévy}}{\implies} (X_t)_{t\geq 0} \text{ is also a d-dim BM starting from } x_0
    \end{align*}

\end{example}
1
\begin{theorem}[Yamada,Watanabe]\label{thm:1.12}
    Let $dX_t=b(X_t)dt+\sigma(X_t)dB_t$ 
    and assume that there exist both a increasing function $\rho(u)\geq 0$ s.t. $|\sigma(x)-\sigma(y)|\leq \rho(|x-y|)\forall x,y\in\R$
    s.t. 
    \[\int_0^\infty \frac{1}{\rho(u)}du=\infty\implies 0<u<C_1\implies \rho(u)\leq C_2 u^{0.5}\]
    and some increasing concave function $\gamma_1(u)\geq 0$ s.t. 
    \[|b(x)-b(y)|\leq \gamma_1(|x-y|)\forall x,y\in\R\] and 
    \[\int_0^\infty\frac{1}{\gamma_1(u)}=\infty.\]
    Then pathwise uniqueness holds.
\end{theorem}

\begin{theorem}[Storokhod]\label{thm::1.13-storokhod}
    Assume that $\sigma,b$ are continuous bounded functions 
    
    $\implies$ there exist weak solutions to the SDE $dX_t=b(X_t)dt+\sigma(X_t)dB_t$.
\end{theorem}

\section{Weak solutions and martingale problems}

Let $(X_t)_{t\geq 0}$ be a weak solution of the SDE \[\begin{cases}
    dX_t&=b(t,X_t)dt+\sigma(t,X_t)dB_t\\X_0&=0
\end{cases}\]
on $(\Omega,\cF,(\cF_t)_{t\geq 0},\bP)$. 

$ \implies X_t$ is a semimartingale s.t. $X_t^k=X_0^k+\int_0^t b(s,X_s)ds+\sum_{l=1}^n\int_0^t \sigma_{k,l}(s,X_s)dB_s^l$
and \[\langle X^i,X^j\rangle_t=\int_0^t \underbrace{\sum_{l=1}^{n}\sigma_{i,l}(s,X_s)\sigma_{j,l}^\intercal(s,X_s)}_{=a_{i,j}(s,X_s)}ds\]

\subsection{Itô-Doeblin formula}

Itô formula leads to 

\begin{proposition}\label{prop:1.14}
    For $f\in C^{1,2}(\R_+\times\R^d)$, then 
    \[f(t,X_t)=f(0,X_0)+\int_0^t(\sigma^\intercal\nabla f)(s,X_s)dB_s+\int_0^t\left[\left(\frac{\partial}{\partial s}+\mathcal{L}\right)\right](s,x_s)ds\]

    where\marginnote{$\mathcal{L}$ is called the \highlight{generator}} $(\mathcal{L}f)(t,x)=\frac{1}{2}\sum_{k,l=1}^{n}a_{k,l}(t,x)\frac{\partial^2}{\partial x_k\partial x_l}f(t,x)+\sum_{k=1}^n b_k(t,x)\frac{\partial}{\partial x_k}f(t,x)$.

\end{proposition}

\begin{remark}
   The Itô-Doeblin formula provides a connection between SDEs and PDEs as we 
   have seen in the last part \dhighlight{Foundations of stochastic analysis}. 
\end{remark}

\begin{example}\label{ex:1.15}
    Let $f\in C^{1,2}(\R_+\times \R_d)$ be a solution of the PDE 
    \begin{align*}
        \frac{\partial}{\partial t}f(t,x)+(\mathcal{L}f)(t,x)&=-g(t,x) & t\geq 0, x\in U\subseteq \R^d\\
        f(t,x)&=\varphi(t,x) & t\geq 0,x\in\partial U.
    \end{align*}
    then $M_t\coloneqq f(t,X_t)+\int_0^t g(s,X_s)ds\in\Mloc$ by proposition \ref{prop:1.14}
    and if $f,g$ are bounded $M_t\in\cM$.

    \[T\coloneqq \inf\{s\geq 0|X_s\notin U\}\implies M_t^T\coloneqq M_{T\land t}\in \mathcal{M}.\]
    Furthermore, if we assume $T<\infty\as$:
    \[\bE[M_t]=\bE[\varphi(T,X_T)]+\bE[\int_0^Tg(s,X_s)ds]=\varphi(0,X_0)\]
    where $X_0=x_0$.

    There are two special cases:

    $g=0\implies$ yields the exit distributions, while 
    
    $\varphi=0,g=1$ yields the mean exit times.
\end{example}

\begin{example}[Feynman-Kac formula]\label{ex1.16:feynman-kac}
    Let $t\in\R_+$ be finite. Assume $f:\R^d\to\R$ and $K:[0,t]\times\R^d\to\R_+$ 
    be continuous functions. Assume that $u$ is a $C^{1,2}$-solution (bounded,for simplicity) of 
    \[\begin{cases}
        \frac{\partial u}{\partial s}(s,x)=\frac{1}{2}\Delta u(s,x)-K(s,x)u(s,x) & s\in [0,t],x\in\R^d\\
        u(0,x)=f(x)& x\in\R^d
    \end{cases}\]
    Then $u$ has the stochastic representation $u(t,x)=\bE_x\left[f(X_t)\exp\left(-\int_0^t K(t-s,X_s)ds\right)\right]$,
    where $X_t$ is BM starting from $X_0=x$.
\end{example}

\begin{proof}[Sketch]
    \begin{enumerate}
        \item Define $r(s,x)\coloneqq u(t-s,x)$ for $s\in[0,t]$
        \item Show: $M_s\coloneqq \exp(-A_s)r(s,x)$ with $A_s=\int_0^s K(u,X_u)du$ is a local martingale.
    \end{enumerate}
\end{proof}

\begin{remark}
    This is a reformulation of the formula from the last semester.
\end{remark}

\subsection{Martingale problem}

A solution of an SDE is generically defined up to some \dhighlight{explosion time $\xi$},
where it either diverges or it exists a given domain $U\subset\R^d$ (open).

$\implies$ For $k\in\N$ define $U_k\coloneqq \{x\in U\mid |x|<k\land \text{dist}(x,U^c)\geq \frac{1}{k}\}$
with $U=\bigcup_{k\geq 1} U_k$ and 
\[T_k\coloneqq \{t\geq 0\mid x_t\notin U_k\}.\]
A solution of the SDE $b(t,X_t)dt+ \sigma(t,X_t)dB_t$ is defined up to $\xi=\sup_{k\geq 1} T_k$.


\beginlecture{03}{18.04.24}

\begin{aremark}
    uniqueness of solution to the heat equation $\frac{1}{2}\Delta u -Ku$: not unique!
    Stochastic calculus: Paolo Baldi: 10.3-10.4
\end{aremark}


Define $(\star)$:

\[dX_t b(t,X_t)dt+\sigma(t,X_t)dB_t\text{ with } X_0=x_0\]

\begin{theorem}[Martingale problem]\label{thm:1.17}
    If $X$ is a solution of $(\star)$ up to time $\zeta$, then $\forall f\in C^{1,2}(\R_+\times U)$
    \[M_t=f(t,X_t)-\int_0^t\left(\frac{\partial}{\partial s}+\mathcal{L}\right)f(s,x_s)ds, t<s\]
    is a local martingale up to $\zeta$ and $M_t^{T_k}$ are localizing martingales.
\end{theorem}

\begin{definition}[Martingale solutions]\label{def:1.18}
    $(X_t)_{t\geq 0}$ is a \dhighlight{martingale solution} of $(\star)$
    if $\forall f\in C^2(\R^d)$
    \[M_t^f=f(X_t)-f(X_0)-\int_0^t(\mathcal{L}f)(X_s)ds\]
    is a continuous local martingale. 
\end{definition}

\begin{theorem}[Equivalent definitions]\label{thm:1.19}
    The following are equivalent (\st{for $X_t$ being a solution of $(\star)$}):
    \begin{enumerate}
        \item[(a)] $\forall f\in C^2(\R^d)$, \[M_t^f\coloneqq f(X_t)-f(X_0)-\int_0^t(\mathcal{L}f)(X_s)ds\] is a local martingale 
        \item[(b)] The process in $\R^d$ given by \[M_t\coloneqq X_t-X_0-\int_0^t b(s,X_s)ds\] is a $d$-dimensional local martingale with $\langle M^i,M^j\rangle_t=\int_0^t a_{i,j}(s,X_s)ds=\langle X^i,X^j\rangle_t$ 
        \item[(c)] $\forall f\in C^{1,2}(\R_+\times \R^d)$ \[\tilde{M}_t^f\coloneqq f(t,X_t)-f(0,X_0)-\int_0^t\left(\frac{\partial}{\partial s}+\mathcal{L}\right)(s,X-s)ds\] is a local martingale  
    \end{enumerate}ds
\end{theorem}

\begin{proof}
    \underline{$c\implies a$}: by choosing $f$ independent of $t$.

    \underline{$ a \implies b$}: 1.: Choosing $f(X)=X_i$ implies\marginnote{We can maybe also proof this by calculating $X^2$?} 
    \begin{align*}
        M_t^f=X_t^i-X_0^i-\int_0^t b_i(s,X_s)ds\in\Mloc
    \end{align*}
    2.: $f(X)=X_iX_j$: 
    \begin{align*}
        (\mathcal{L}f)(x)&=\frac{1}{2}a_{ij}+\frac{1}{2}a_{ji}+b_iX_j b_jX_i\\
        &a=a^\intercal\implies =a_{ij} b_iX_j b_j X_i\\
        \implies& M_t^f X_t^i X_t^j-X_0^iX_0^j-\int_0^t\left[a_{ij}(s,X_s)+b_i(s,X_s)X_s^j+b_j(s,X_s)X_s^i\right]ds
    \end{align*}
    \marginnote{Here $dX_s^i$ is the same as $b_iX_s^j$ is the same up to a local martingale term and $\langle X^i,X^j\rangle_t=\int_0^t a_{ij}(s,X_s)ds=\langle M^i,M^j\rangle_t$} 
    \begin{align*}
        X_t^iX_t^j-X_0^iX_0^j&\stackrel{\text{Integration by parts}}{=}\int_0^t X_s^i dX_s^j+\int_0^t X-s^j dX_s^i + \langle X^i,X^j\rangle_t\\
        &= M_t^f+\int_0^t [a_{ij}+b_iX_s^j+b_jX_s^i] ds
    \end{align*}


    \underline{$b\implies c$}: By Proposition \ref{prop:1.14} If (use the next theorem) $X$ was a weak solution $\implies \tilde{M}_t^f$ is a local martingale. 
\end{proof}


\begin{theorem}\label{thm:1.20}
    Let $n=d$\marginnote{This also works for $n\neq d$, but with a different proof}, % FIX
    assume $\sigma(t,x)$ is invertible
    $\forall t,x$ and $\sigma^{-1}(t,x)$ is uniformly bounded. T.f.a.e.:
    \begin{enumerate}
        \item[(a)] $(X_t)_{t\geq 0}$ is a weak solution of the SDE $(\star)$ on $(\Omega,\cF,(\cF_t)_{t\geq 0},\mathbb{P};B)$
        \item[(b)] $(X_t)_{t\geq 0}$ is a martingale solution of the SDE $(\star)$ 
    \end{enumerate}
\end{theorem}

\begin{proof}
    $a\implies b$: True 

    $b\implies a$: Goal construct a BM for the weak solution. 

    By proposition \ref{prop:1.19} $a\implies b$ $dM_t=dX_t-b(t,X_t)dt\in\Mloc$ and $d\langle M^i,M^j\rangle_t={a_{k,l}}(t,X_t)dt$

    \begin{align*}
        \implies dX_t&=dM_t+b(t,X_t)dt\\
        &=\sigma(t,X_t)d\tilde{B}_t+b(t,X-t)dt\\
    \end{align*}
    where $\tilde{B}_t\coloneqq \sigma(s,X_s)^{-1}dM_s$

    To see: $\tilde{B}_t$ is a brownian motion.

    \begin{align*}
        \langle \tilde{B}^i,\tilde{B}^j\rangle_t&=\sum_{k,l}\int_0^t\sigma_{ij}^{-1}\sigma_{jl}^{-1} \underbrace{d\langle M^k,M^l\rangle_s}_{=\underbrace{a_{ij}}_{(\sigma^\intercal \sigma)_{kl}}ds}\\
        &=\sum_{k,l,p}\int_0^t\sigma_{ik}^{-\intercal}\sigma_{kp}\sigma_{pl}^{\intercal}\sigma_{lj}^{-\intercal}ds\\
        &=\delta_{ij}\int_0^t 1 ds=\delta_{ij}t
    \end{align*}
    Then by the Lévy characterization $\tilde{B}$ is a brownian motion.
\end{proof}

\begin{aremark}
    This is the first way to construct a weak solution: Solve a martingale problem! This is used a lot in practice.
\end{aremark}

\section{Weak solutions and time change}

\subsection{Time change}

For $d=1$:

\begin{theorem}\label{thm:1.21}[Dubins-Schwarz]
    \begin{itemize}
        \item Let $M\in\Mloc^0$ and $\langle M\rangle_\infty=\infty$ a.s. 
        \item Let $T_t\coloneqq \inf\{s\geq |\langle M\rangle_s \geq t\}$
    \end{itemize}
    This implies    
    \begin{enumerate}
        \item $t\mapsto M_{T_t}$ os a $(\cF_{T_t})$ brownian motion
        \item $M_t=B_{\langle M\rangle_t}$ for some standard brownian motion $B$
    \end{enumerate}
\end{theorem}

$X_t=X_0+\int_0^t b(s,X_s)ds+\underbrace{\int_0^t \sigma(s,X_s)dB_s}_{=M_t}$.
If $\langle M\rangle_\infty=\infty$ a.s.:
\[X_t=X_0+\int_0^tb(s,X_s)ds+\tilde{B}_{\int_0^t \sigma^2(s,X_s)ds}\]

\subsection{Time change in a martingale problem}

Consider $d=1=n$.

\[dY_t=\tilde{\sigma}(Y_t)dB_t \hspace{3cm}(\star\star)\]
and $\tilde{\sigma}$ strictly positive positive continuous function.

\begin{align*}
    \langle Y\rangle_t=\int_0^t\tilde{\sigma}(Y_s)^2 ds\eqqcolon A_t
\end{align*}
By theorem \ref{thm:1.21} $\implies Y_t=W_{A_t}$ for some brownian motion $W$.

Assume $A_t\infty=\infty$ a.s.

$T_t\coloneqq \inf\{s\geq 0\mid \langle Y\rangle_s \geq t\}$

\[\implies T_{A_t}=\inf\{s\geq 0\mid \langle Y\rangle_s\geq \langle Y\rangle_t\}=t\]

\begin{align*}
    1&=\frac{d}{dt}\left(T_{A_t}\right)=T_{\underbrace{A_t}_{=u}}'\cdot A_t\\
    \implies \underbrace{T_u'}_{=\frac{dT_u}{du}}&=\frac{1}{A_{T_u}'} \implies T_u = \int_0^u \frac{1}{\tilde{\sigma}(Y_{T_s})^2}ds=\int_0^u \frac{1}{\tilde{\sigma}(W_s)^2}ds
\end{align*}

$\implies$ to construct a solution of $(\star\star)$: Given $W$ $\longrightarrow$ compute $T_u$ $\longrightarrow$ determine $A-t=T_t^{-1}\implies Y_t=W_{A_t}$

\beginlecture{04}{23-04-24}

\begin{theorem}\label{thm:1.22}
    Let $(X_u)_{u\geq 0}$ on $(\Omega,\cF,(\cF_t)_{t\geq 0},\bP)$ be a weak solution of 
    \[dX_u=b(X_u)du+\sigma(X_u)dB_u\]
    where $b:\R^d\to\R^d$,the drift and $\sigma:\R^d\to\R^d\times\R^d$ are locally bounded, 
    $\sigma^{-1}$ exists for a.e. $x$ and is locally bounded.
    
    Consider a \dhighlight{time change} $T_u\coloneqq \int_0^u \rho(X_s)ds$,
    where $\rho:\R^d\to\R_+$ s.t. 
    \[T_u<\infty \forall u\geq 0\text{ and } T_\infty=\infty\as\]

    $\implies$ Then $Y_t\coloneqq X_{A_t}$, where $A_t=T_t^{-1}$ is a weak 
    solution of the SDE 
    \[dY_t=\frac{b(Y_t)}{\rho(Y_t)}dt+\frac{\sigma(Y_t)}{\sqrt{\rho(Y_t)}}dB_t\]
\end{theorem}

\begin{remark}
    Special case: $d=1,b=0,\sigma=1$: Then $X$ is a BM and $\rho=\frac{1}{\tilde{\sigma}^2(x)}\implies Y_t=X_{T_t^{-1}}$ solves 
    \[dY_t=\tilde{\sigma}(Y_t)dB_t\]
\end{remark}

\begin{proof}
    By theorem \ref{thm:1.20}, it is enough to see how the martingale problem after 
    change of time becomes: 
    \[\mathcal{L}=\mathcal{L}_{X_t}\stackrel{\text{time change}}{\longrightarrow}\mathcal{L}_{Y_t}=\tilde{\mathcal{L}}\]
    $Y_t=X_{A_t};Y_0=X_{A_0}$. For $f\in C^2: M_t^f=f(X_t)-f(X_0)-\int_0^t (\mathcal{L}f)(X_s)ds$ is a local martingale w.r.t. $(\cF_t)_{t\geq 0}$.
    \marginnote{The $A_0$ in the integral is probably $0$, but it does not matter, we do a change of variables anyway.}\[\implies N_t^f\coloneqq M_{A_t}^f=f(\underbrace{X_{A_t}}_{=Y_t})-f(\underbrace{=X_{A_0}}_{Y_0})-\int_{A_0}^{A_t}(\mathcal{L}f)(X_s)ds\]
    is also a local martingale w.r.t. $(\cF_{A_t})_{t\geq 0}$.

    Change of variable (to get rid of the $X_s$ in the integral):
    \begin{align*}
        \tau&=T_s\leftrightarrow A_t=s \implies X_s=X_{A_t}=Y_\tau \\
        d\tau&=\rho(X_s)ds
    \end{align*}

    \begin{align*}
        \implies N_t^f=f(Y_t)-f(Y_0)-\int_0^t(\mathcal{L}f)(Y_t)\frac{1}{\rho(Y_t)}d\tau
    \end{align*} 
    Since $\mathcal{L}f(x)=\sum_{k}b_k(x)\frac{\partial}{\partial x_k}f(x)+\sum_{k,l}a_{k,l}(x)\frac{\partial^2}{\partial x_k\partial x_l} f(x)$

    \[\implies (\tilde{\mathcal{L}}f)(x)=\sum_{k}\frac{b_k(x)}{\rho(x)}\frac{\partial}{\partial x_k}f(x)+\sum_{k,l}\frac{\overbrace{a_{k,l}}^{(\sigma\sigma^\intercal)_{k,l}}}{\sqrt{{\rho(x)}{\rho(x)}}}(x)\frac{\partial^2}{\partial x_k\partial x_l} f(x)\]

    $\implies$ It is a martingale problem for the SDE 
    where the $\text{drift}\to\frac{\text{drift}}{\rho}$ and $\sigma\to \frac{\sigma}{\sqrt{\rho}}$

\end{proof}

\subsection{Weak solutions in d=1}

We will do both time and ``space'' changes. 
\begin{itemize}
    \item 1-d SDE:\begin{equation}\label{eq:1.4}\begin{cases}
        dX_t&=b(X_t)dt+\sigma(X_t)dB_t\\
        X_0=x_0\in(\alpha,\beta)
    \end{cases}\end{equation}
    \item  $X_t$ a process in $(\alpha,\beta)$
    \item Assume $b,\sigma:(\alpha,\beta)\to\R$ continuous, $\sigma(x)>0\forall x\in (\alpha,\beta)$
    \item Do a change of coordinates $Y_t\coloneqq s(X_t)$ where $s:(\alpha,\beta)\to(s(\alpha),s(\beta))$, $C^2$ with $S'(x)>0,x\in (\alpha,\beta)$.
    \item $s(x)$ is called the \dhighlight{scale function} and it is given by \[s(x)=\int_{x_0}^x\exp\left(-\int_{x_0}^{y}dz\frac{2b(z)}{\sigma(z)^2}\right)dy\]
    \item $s$ satisfies $\frac{1}{2}\sigma^2(x)s''(x)+b(x)s'(x)=0\iff (\mathcal{L}s)(x)=0$ 
\end{itemize}

\begin{remark}
    If $b(z)=0\implies s(x)=x-x_0\implies s'(x)=1$. If $s'(x)=1$, we say that the process is in its ``natural scale''
\end{remark}

By proposition \ref{prop:1.14}: $\mathcal{L}s=0,\dot{s}=0$.

$\implies Y_t=s(X_t)$ is a local martingale satisfies $dY_t=s'(X_t)\sigma(X_t)dB_t$ \marginnote{the other terms cancel}.  

$\iff Y_t$ is a solution of 

\begin{equation}\label{eq:1.5}
    \begin{cases}
        dY_t&=\tilde{\sigma}(Y_t)dB_t\\
        Y_0&=s(X_0)
    \end{cases}
\end{equation}
where $\tilde{\sigma}(y)=s'(s^{-1}s(y))\sigma(s^{-1}(y))$.

\marginnote{Therefore we can write the original SDE in terms of a BM}

\begin{theorem}\label{thm:1.23:weak_solution_1d}
    The following are equivalent:
    \begin{enumerate}
        \item The process $(X_t)_{t<\xi}$, where $\xi$ is the explosion time, on $(\Omega,\cF,(\cF_t)_{t\geq 0},\bP;(B_t)_{t\geq 0})$ is a solution of (\ref{eq:1.4}) up to tje stopping time $\xi$
        \item The process $Y_t=s(X_t)_{t<\xi}$\marginnote{$s$ and $A_t$ have the same definition as before} on $(\Omega,\cF,(\cF_t)_{t\geq 0},\bP;(B_t)_{t\geq 0})$ is a solution of (\ref{eq:1.5}) up to $\xi$
        \item The process $(Y_t)_{t<\xi}$ has the representation $Y_t=\tilde{B}_{A_t}$, where $\tilde{B}$ is a BM starting at $\tilde{B}_0=s(X_0)$ and $A_t=T_t^{-1}$ and $T_t=\int_0^t\frac{1}{\tilde{\sigma}^2(\tilde{B}_u)}du$
    \end{enumerate}    
\end{theorem}


\dhighlight{\large{A degenerate case:}}

Let $\sigma(x)=|x|^\alpha$ for some $\alpha\in \left(0,\frac{1}{2}\right)$.
$\implies$
\begin{equation}\label{eq:1.6}
    \begin{cases}
        dY_t &= |Y_t|^\alpha dB_t\\
        Y_0&=y
    \end{cases}
\end{equation}

$\implies T_t=\int_0^t\frac{1}{\sigma(\tilde{B}_u)^2}du$, $A_t=\int_0^t\sigma(Y_s)^2ds$ and $Y_t=\tilde{B}_{A_t}\implies \tilde{B}_0=y$.

$T_t<\infty\as$?
\begin{align*}
    \bE(T_t)&=\int_0^t \bE\left(\frac{1}{\sigma(\tilde{B})^2}\right)du\\
    &=\int_0^t \bE\left(\frac{1}{|\tilde{B}|^{2\alpha}}\right)du\\
    &=\int_0^t\int_\R dx \frac{1}{|x|^{2\alpha}}\frac{\exp(-\frac{(x-y)^2}{2u})}{\sqrt{2\pi u}}\stackrel{0<2\alpha<1}{<}\infty
\end{align*}

$\implies A_t=T_t^{-1}$, then $Y_t=\tilde{B}_{A_t}$ is a solution of (\ref{eq:1.6}),
i.e. $\forall y\in\R\exists$ a non-trivial solution of (\ref{eq:1.6}).

For $Y_=0,Y_t=0$ is also a solution $\implies$
\begin{itemize}
    \item No uniqueness in law of (\ref{eq:1.6})
    \item No pathwise uniqueness as well
\end{itemize}

\begin{remark}
    In general: uniqueness in law of $1$-d SDEs is not to be expected if 
    $\sigma(x)=0$ somewhere (and $\sigma$ continuous \dots) (i.e. if $\sigma$ is degenerate). 
\end{remark}

By theorem \ref{thm:1.12} as soon as $\sigma(x)=|x|^\alpha$ for some $\alpha\geq \frac{1}{2}$,
then one has pathwise uniqueness.

\dhighlight{\large{Hitting times and scale functions}}
\highlight{Bessel process}: 
\[\begin{cases}
    dR_t&=\frac{d-1}{2R_t}dt+dW_t\\
    R_0 &=r_0\in (0,\infty)\end{cases}\] 
    
$\implies b(x)=\frac{d-1}{2x},\sigma(x)=1$. 

The scale function satisfies $\mathcal{L}s(x)=\frac{1}{2}s''(x)+\frac{d-1}{2x}s'(x)=0$

\[\implies s(x)=\begin{cases}
    \frac{x^{2-d}}{2-d} & d\neq 2\\
    \ln(x) & d=2
\end{cases}\]

and therefore 

\[(s_0,s_\infty)=\begin{cases}
    (0,\infty) & d<2\\
    (-\infty,\infty) & d=2\\
    (-\infty,0 & d>2)
\end{cases}\]

Define the stopping time
\[T_a^R\coloneqq \inf\{t\geq 0\mid R_t=a\}\]

Choose an $\alpha<r_0<\beta$

\[\implies \bP(T_\alpha^R<T_\beta^R)\stackrel{s'>0}{=}\bP(T_{s(\alpha)}^{s(R)}<T_{s(\beta)}^{s(R)})\]

One computes\marginnote{WS exercises}
\begin{align*}
    \bP(T_a^{R}<T_\beta^R)=\frac{s(\beta)-s(r_0)}{s(\beta)-s(\alpha)}=\bP(T_{s(\alpha)}^{s(R)}<T_{s(\beta)}^{s(R)})
\end{align*}

This is generic, for Itô diffusions, provides that $\exists$ no killing\marginnote{unlike in \ref{ex1.16:feynman-kac}} in $(\alpha,\beta)$.

\beginlecture{05}{25.04.24}

\subsection{Uniqueness of martingale solution}

SDE $dX_t=b(t,X_t)dt+\sigma(t,X_t)dB_t$ with generator 
\[\mathcal{L}=\sum_{k}b_k\frac{\partial}{\partial x_k}+\frac{1}{2}\sum_{k,l}(\sigma\sigma^\intercal)_{k,l}\frac{\partial^2}{\partial x_k\partial x_l}\]

\begin{definition}\label{def:1.24:martingale_solution}
    Let $\mathcal{C}=C(\R_+,\R^d)$ with $\sigma$-algebra $\cF$,
    canonical filtration $(\cF_t)_{t\geq 0}$, canonical process $Z_t(\omega)\coloneqq \omega$.

    We say that $\bP$ on $(\cC,\cF)$ is a \dhighlight{martingale solution} for the generator $\cL$ 
    $\iff \forall f\in C^{1,2}(\R_+\times \R^d\to\R)$  
    \begin{equation}\label{eq:1.7}
        M_t^f\coloneqq f(t,Z_t)-f(0,Z_0)-\int_0^t\left( \frac{\partial}{\partial s}+\cL\right)f(s,Z_s)ds 
    \end{equation}
    is a martingale w.r.t. $\bP$.
\end{definition}

\begin{definition}\label{def:1.25}
    A martingale problem (\ref{eq:1.7}) has a unique solution if for any two 
    martingale solutions $\bP=\Q$ s.t. $\text{Law}_{\bP}(Z_0)=\text{Law}_{\Q}(Z_0)$
    \[\implies \bP=\Q\]
\end{definition}

\begin{remark}
    Uniqueness of martingale solutions corresponds to uniqueness in law of the weak solutions.
\end{remark}

Backwards Kolmogorov Equation (BKE):

\begin{equation}\label{eq:1.8:BKE}
    \frac{\partial}{\partial t}\varphi(t,x)=\cL\varphi(t,x)\forall x\in\R^d,t\geq 0
\end{equation}

\begin{theorem}\label{thm:1.26}\marginnote{The Kolmogorov forward equation is (related to) the Folker-Plank equation!}
    Assume that $\forall$ initial condition 
    \[\varphi(0,x)=\Psi(x),\Psi\in C_0^\infty(\R^d)\]
    the (\ref{eq:1.8:BKE}) has a solution and $\varphi$ bounded for all finite time intervals. We have uniqueness of martingale solutions and therefore uniqueness of weak solutions!
\end{theorem}

\begin{proof}
    Prove that $\forall 0\leq t_1<t_2<\dots<t_n$:
    \[\text{Law}_{\bP}(Z_{t_1},Z_{t_2},\dots,Z_{t_n})=\text{Law}_{\Q}(Z_{t_1},Z_{t_2},\dots,Z_{t_n})\]

    \highlight{1. One-time distribution:}

    $\forall 0\leq s\leq r$: 
    \begin{align*}
        \left(\frac{\partial}{\partial s}+\cL\right)\varphi(r-s,x)\stackrel{(\ref{eq:1.8:BKE})}{=}0
    \end{align*}

    take $t\in[0,r]$: \marginnote{This needs the boundedness of $\varphi$, otherwise it might only be a local martingale. There are softer contidions we can put on the coefficients to achieve the same result. This might not be needed, because 
                        $\varphi$ is $C^1$ in time anyways}
    \begin{align*}
        M_t^r&\coloneqq \varphi(r-t,Z_t)-\varphi(r,Z_0)-\int_0^t\underbrace{(\partial_s+\cL)\varphi(r-s,Z_s)}_{=0}ds\\
        &=\varphi(r-t,Z_t)-\varphi(r,Z_0)\text{ is a martingale}
    \end{align*}
    for any solution $\bP$. 

    \begin{eqnarray*}
        0&=\bE_{\bP}\left(M_r^r-M_t^r\mid\cF_t \right) &= \bE(\varphi(0,Z_r)-\varphi(r-t,Z_t)\mid \cF_t)\\
        \implies &\forall 0\leq t\leq r: \bE_\bP(\varphi(0,Z_r)\mid\cF_t)&=\bE_\bP(\varphi(r-t,Z_t)\mid\cF_t)\\
        &&\stackrel{\as}{=}\varphi(r-t,Z_t)\\
        &\bE(\underbrace{\varphi(0,Z_r)}_{\Psi(Z_r)})&\stackrel{t=0}{=}\bE_bP(\varphi(r,Z_0))
    \end{eqnarray*}

    $\forall $ other martingale solutions $\Q$:

    \begin{eqnarray*}
        \bE_\Q(\Psi(Z_r))=\bE_{\Q}(\varphi(r,Z_0))
    \end{eqnarray*}

    By assumption this implies $\text{Law}_\bP(Z_r)=\text{Law}_\Q(Z_r)$.

    \highlight{2. Multi-time distributions:}

    For $\Psi\in C_0^\infty$, denote $\varphi_\Psi$ the solution of (\ref{eq:1.8:BKE}) with initial condition $\Psi$:
    \begin{eqnarray*}
        \bE_\bP(\Psi(Z_r)\mid \cF_t)=\varphi_\Psi(r-t,Z_t)
    \end{eqnarray*}

    $0\leq r_2\leq r_1$ test for $g\in C_0^\infty$:
    \begin{align*}
        \bE_\bP(\Psi(Z_{r_1})g(X_{r_2}))&=\bE(\underbrace{\bE(\varphi_\Psi(Z_{r_1})|\cF_{r_2})}_{\varphi_\Psi(r_1-r_2,Z_{r_2})}g(Z_{r_2}))\\
        &=\bE_\bP(\varphi_\Psi(r_1-r_2,Z_{r_2})g(Z_{r_2}))\\
        &\stackrel{1.}{=}\bE_\Q(\varphi_\Psi(r_1-r_2,Z_{r_2})g(Z_{r_2}))\\
        &=\bE_\Q(\Psi(Z_{r_1})g(Z_{r_2}))
    \end{align*}

    Iterating yields the statement.

\end{proof}
