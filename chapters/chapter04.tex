
\chapter{Levy processes}\beginlecture{14}{11.06.24}

\section{Basics}

Poisson process (Picture) ... 
With $t_{n+1}-t_{n}$ are iid., $\sim\exp(\lambda)$ (here it is the parameter, NOT the mean!) and $N_t\sim \Poi(\lambda,t)$.

\[\mathbb{P}(N_{t+\epsilon}=y+1|N_t=y)=\lambda\epsilon+O(\epsilon^2)\]

\begin{center} % TODO: FIX
    \begin{tabular}{|c|c|}\hline
        BM  & Poisson process\\\hline
        & Monotone increasing\\\hline
    \end{tabular}
\end{center}

BM:

A real valued process $B=(B_t)_t$ satisfying:\begin{enumerate}
    \item[(a)] $t\mapsto B_t$ is continuous
    \item[(b)] $\bP(B_0=0)=1$
    \item[(c)] For $0\leq s\leq t: B_t-B_s\stackrel{d}{=}B_{t-s}$
    \item[(d)] For $0\leq s\leq t: B_t-B_s$ is independent of $\{B_u,u\leq s\}$
    \item[(e)] $\forall t>0B_t\sim \mathcal{N}(0,t)$   
\end{enumerate}

Poisson process: 

A process $N=(N_t)$ with values in $D$ is called a Poisson process with parameter $\lambda>0$, if 
\begin{enumerate}
    \item[(a)] $t\mapsto N_t$ is $\mathbb{P}$. a.s. \dhighlight{cadlag}.
    \item[(b)] $\bP(N_0=0)=1$
    \item[(c)] For $0\leq s\leq t: N_t-N_s\stackrel{d}{=}M_{t-s}$
    \item[(d)] For $0\leq s\leq t: N_t-N_s$ is independent of $\{N_u,u\leq s\}$
    \item[(e)] $\forall t>0N_t\sim \Poi(\lambda t)$   
\end{enumerate}

Notice how b,c,d are the same!

Differences:

BM: 
\begin{itemize}
    \item continuous paths 
    \item unbounded variation
\end{itemize}

PP: 
\begin{itemize}
    \item pure jump process
    \item bounded variation
\end{itemize}

Common features: 

\begin{itemize}
    \item both have càdlàg trajectories
    \item stationary, independent increments
\end{itemize}

\begin{definition}[Lévy Process]\label{def:4.1}
    A process $X=(X_t)_{t\geq 0}$ defined on $(\Omega,\cF,\bP)$ is a \dhighlight{Lévy process}, if:
    \begin{enumerate}
        \item[(a)] The trajectories are $\bP$. a.s. càdlàg 
        \item[(b)] $\bP(X_0=0)=1$
        \item[(c)] For $0\leq s\leq t: X_t-X_s\stackrel{d}{=}X_{t-s}$
        \item[(d)] For $0\leq s\leq t: X_t-X_s$ is independent of $\{X_u,u\leq s\}$.    
    \end{enumerate}
\end{definition}

\begin{remark}
    Lévy processes are Markov processes.
\end{remark}

\begin{definition}\label{def:4.2}
    A real-valued random variable $X$ has a \dhighlight{infinitely divisible distribution}, if 
    for all $n\geq 1$: There exists r.v. $X_1^n,\dots,X_n^n$ iid. s.t. \[X\stackrel{d}{=}X_1^n+\dots X_n^n.\] 
    Let $X\sim \mu: \mu$ \dhighlight{is infinitely  divisible} if for all $n\geq 1$:
    $\exists$ prob. measure $\mu_n$ s.t. $\mu=(\mu_n)^{*n}$, i.e. the $n$-fold convolution. 
\end{definition}

\begin{remark}\marginnote{We can change the definition to $\bE(e^{iuX})=e^{-\psi(u)}$ to avoid taking logarithms of complex numbers.}
    Setting $\psi(u)=-\log(\bE(e^{iuX}))$, for $u\in \R$, $\implies \psi$ is called the \dhighlight{charateristic exponent of $X$}.
    If $X$ is infinitely divisible, then 
    \begin{align*}
        \bE(e^{iuX})&=\bE(e^{iu(X_1^n+\dots+X_n^b)})\\
        &=\bE(e^{iuX})^n \\
        &\implies \psi(u)=n\psi_n(u),
    \end{align*} 
    where $\psi_n$ is the characteristic exponent of $X_1^n$.
\end{remark}

\begin{theorem}[Lévy-Khintchine formula]\label{thm:4.2}
    A probability law $\mu$ of real valued real random variable is infinitely divisible with
    characteristic exponent
    \[\psi(u\theta)=-\log\int_\R e^{i\theta x}\mu(dx)\]
    if and only if $\exists$ a triple $(a,\sigma,\nu)$, $a\in\R,\sigma>0$,$\mu$ a measure supported on \marginnote{$\nu$ does not have to have finite measure!}
    $\R\setminus\{0\}$ satisfying $\int_\R (1\land x^2)\nu d(x)<\infty$ s.t.
    \[\psi(\theta)=ia\theta+\frac{1}{2}\theta^2\sigma^2+\int_R(1-e^{i\theta x}+i\theta x 1_{|x|< 1})\nu(dx)\] 
\end{theorem}

\begin{aremark}
    $\nu$ is called the \dhighlight{Lévy measure}.
\end{aremark}

\begin{aremark}
    What happens for $x$ small? 
    \[(1-e^{i\theta x}-i\theta x 1_{|x|< 1})\approx 1- (1+i\theta x-\frac{\theta^2 x^2}{2})+i\theta x=\frac{\theta^2 x^2}{2}\]
\end{aremark}

\begin{aremark}
    The 1 in the indicator function is arbitrary, tho other choices yield other formulas!
\end{aremark}

\subsection{Relation of infinitely divisible processes with Levy processes}

Let $X$ be a Levy process, $\forall t>0$, $X_t$ is infinitely divisible, since 
\[(\star) \text{ }X_{t}=X_{\frac{t}{n}}+\left(X_{\frac{2t}{n}}-X_{\frac{t}{n}}\right)+\dots+\left(X_{t}-X_{\frac{t(n-1)}{n}}\right)\]
For all $\theta\in\R,\geq 0$: $\psi_t(\theta)\coloneqq -\ln(\bE(e^{i\theta X_t}))$
\[\stackrel{(\star)}{\implies}\forall m,n\geq 1: \frac{m}{m}\psi_1(\theta)=\frac{1}{m}\psi_m(\theta)=\frac{n}{m}\psi_{\frac{m}{n}}(\theta)\]
and therefore for $t\geq 0$ rational: $\psi_t(\theta)=t\psi_1(\theta)$.
It then follows that $\forall t\in\R_+: \psi_t(\theta)=t\psi_1(\theta)$.
For Lévy process:
\[\bE(e^{i\theta X_t})=e^{-t\psi_1(\theta)}\]
\begin{definition}\label{def:4.3}
    We refer to $\psi(\theta)\coloneqq \psi_1(\theta)$ to be the characteristic exponent of the Lévy process.
\end{definition}

\begin{theorem}\label{thm:4.4}
    Let $a\in\R,\sigma\geq 0,\nu$ a measure supported on $\R\setminus\{0\}$ s.t. \[\int_\R (1\land x^2)\nu(dx)<\infty\]
    Define \[\psi(\theta)\coloneqq ia\theta+\frac{1}{2}\sigma^2\theta^2+\int_\R (1-e^{i\theta x}+i\theta x)1_{|x|<1}\nu(dx)\]
    $\implies$ then there exists a Lévy process with characteristic exponent $\psi(\theta)$.
\end{theorem}

\begin{proposition}\label{Prop:4.5}
    Let $F$ be a distribution function on $\R_+$. Then $F$ is the distribution function of an infinitely 
    divisible law
    
    $\iff$ $\forall\lambda >0: F^\mathcal{L}(\lambda)=\int_0^\infty e^{-\lambda x}F(dx)=e^{-c\lambda-\int_0^\infty (1-e^{-\lambda x})\mu(dx)}$,
    where $c\in\R$, $\mu$ is a measure on $(0,\infty)$, s.t. $\int_0^\infty (1\land x)\mu(dx)<\infty$. 
\end{proposition}


\begin{proof}
    $\implies:$ Let $X$ have distribution $F:F(x)=\bP(X\leq x)$. $X_1^n+\dots+X_n^n$ with $\bP(X_1^n\leq x)\eqqcolon F_n(x)$.   
    \[F^{\mathcal{L}}(\lambda)=\int_0^\infty e^{-\lambda x}F(dx)=(F_n^{\mathcal{L}}(\lambda))^n,\]
    since 
    \begin{align*}
        F^{\mathcal{L}}(\lambda)&=\int_0^\infty e^{-\lambda x}\mathbb{P}(X_1+\dots+X_n\in dx)\\
        &=\int_0^\infty e^{-\lambda x}\int \bP(X_1\in dz)\bP(X_2+\dots+X_n+z\in dx)\\
        &=\int_0^\infty \bP(X_1\in dz)\int_z^\infty \bP(X_2+\dots+X_n+z\in dx)e^{-\lambda x}\\
        &\stackrel{x=\tilde{x}+z}{=}\int_0^\infty \bP(X_2+\dots+X_n\in d\tilde{x})e^{-\lambda \tilde{x}}e^{-\lambda z}\\
        &=\left(\int_0^\infty \bP(X_1\in dz)e^{-\lambda z}\right)^n = \left(F_n^{\mathcal{L}}(\lambda)\right)^n
    \end{align*}

    Since $F$ is a distribution on $\R_+\implies F_n^{\mathcal{L}}\in (0,1]$, which means 
    \[F_n^{\mathcal{L}}(\lambda)\stackrel{n\to\infty}{\to}1\]
    uniformly for $\lambda$ in compact sets. \marginnote{This last inequality is almost an equality for large $n$!}
    \[\implies \ln F^{\mathcal{L}}(\lambda)=n\log F_n^{\mathcal{L}}(\lambda)=n\log (1-(1-F_n^{\mathcal{L}}(\lambda)))\leq -n (1-F_n^{\mathcal{L}}(\lambda))\]

    1.: $\forall \delta>0,K<\infty,\exists n_0<\infty$ s.t. $n\geq n_0$, $\lambda \leq k\implies 1-F_n^{\mathcal{L}}(\lambda)\leq \delta$.

    2.: $\forall \delta>0, C<\infty$ s.t. $\forall 0\leq x\leq \delta$,
    \[-x(1+cx)\leq \log(1-x)\leq -x\]

    Apply it to $x=1-F_n^{\mathcal{L}}(\lambda)$
    \[\implies -(1-F_n^{\mathcal{L}}(\lambda))[1+C(1-F_n^{\mathcal{L}}(\lambda))]\leq \log F_n^{\mathcal{L}}(\lambda)-n(1-F_n^{\mathcal{L}}(\lambda))\]

    As $n\to\infty$, uniformly in bounded set (for $\lambda$):
    \[-n(1-F_n^{\mathcal{L}}(\lambda))\to\log F^{\mathcal{L}}(\lambda).\]

    \begin{align*}
        n(1-F_n^{\mathcal{L}}(\lambda))&=\int(1-e^{-\lambda x}) F_n(dx)\\
        &= \int\frac{1-e^{-\lambda x}}{1-e^{-x}} \underbrace{n(1-e^{-x}) F_n(dx)}_{\eqcolon m_n(dx)}
    \end{align*}

    $m_n$ is a measure on $(0,\infty)$ with total mass is $n(1-F_n^{\mathcal{L}}(1))\to\log F^{\mathcal{L}}(1)$

    $\implies\exists$ measure $m$ on $[0,\infty]$ s.t. $m_n\to m$ and 
    \[n(1-F_n^{\mathcal{L}}(\lambda))\to m(\{0\})\lambda+\int_0^\infty \frac{1-e^{-\lambda x}}{1-e^{-x}}m(dx)+m(\{0\})\]
    Let $\lambda=0$: $m(\{\infty\})=-\log F^{\mathcal{L}}(0)=-\log 1=0$. Then with $c=m(\{0\})$
    \[\mu(dx)(1-e^{-x})=m(dx)\]

    $\impliedby$ ... 
\end{proof}


\beginlecture{15}{13.06.24}


Question: What are $a,\sigma,\nu$? Is there a relation between $\nu$ and the jumps of the Lévy process?

\section{Examples}

\subsection{Poisson process}

Let $\lambda>0$ and $X\sim\Poi(\lambda)$,
\begin{align*}
    \bE(e^{i\theta X})&=\sum_{n\geq 0}e^{-\theta n}\frac{e^{-\lambda}\lambda^n}{n!}\\
    &=e^{-\lambda}e^{\lambda e^{i\theta}}\\
    &=e^{-\lambda(1-e^{i\theta})}=\left(e^{-\frac{\lambda}{n}(1-e^{i\theta})}\right)^n
\end{align*}

$\implies X$ is infinitely divisible.

To get the right characteristic exponent \[-\log\bE(e^{i\theta X})=e^{-\lambda}e^{\lambda e^{i\theta}}\]
which means $\psi(  \theta)=\lambda(1-e^{i\theta})$.

Take $a=0,\sigma=0,\nu(dx)=\lambda\delta_1(x)$, then $\psi(\theta)=(1-e^{i\theta}\lambda)$.

The Lévy process with characteristic exponent $\psi(\theta)$ is called the Poisson process with parameter $\lambda>0$.

\begin{aremark}
    If you have a Lévy process on $\mathbb{Z}_{\geq 0}$, $(X_t)$ with $X_0=0$ and 
    with jumps $+1\implies$ This is a Poisson process.
\end{aremark}

\subsection{Compound Poisson process}

Let $N$ be a Poisson random variable with parameter $\lambda>0$, $\{\xi_k\}_{k\geq 1}$ iid r.v. (independent of $N$)\marginnote{The $\xi$ give us the size of the jumps}
with distribution $F$ on $\R\setminus\{0\}$.

Let $X=\coloneqq \sum_{k=1}^N\xi_k$:
\begin{align*}
    \implies \bE\left(e^{i\theta X}\right)&=\sum_{n\geq 0}\bE\left(e^{i\theta(\xi_1+\dots+\xi_n)}\right)\frac{e^{-\lambda \lambda^n}}{n!}\\
    &=\sum_{n\geq 0}\frac{e^{-\lambda} \lambda^n}{n!}\left(\int e^{i\theta X}F(dx)\right)^n \\
    &=e^{-\lambda}e^{\lambda\int e^{i\theta X}F(dx)}=e^{-\lambda \int F(dx)}e^{\lambda\int e^{i\theta X}F(dx)}\\
    &=e^{-\underbrace{\lambda\int (1-e^{i\theta X})F(dx)}_{=\psi(\theta)}}
\end{align*}

$\implies$ $X$ is infinitely divisible with $\psi(\theta)$ having $\sigma=0,\nu(dx)=\lambda F(dx),a=-\int{|x|<1} x F(dx)$

How to get the process?
\marginnote{We can think of a poisson process sa the set of jumping points (times). This is sometimes called poisson jump process}
\begin{enumerate}
    \item Choose Poisson process with parameter $\lambda$, $(N_t)_{t\geq 0}$
    \item $X_t=\sum_{k=1}^{N_t}\xi_k$ is called the \dhighlight{compound Poisson process}
\end{enumerate}

Is $(X_t)_{t\geq 0}$ a Lévy process?\marginnote{Notation $0\eqqcolon \sum_{k=m}^n (\dots)$ for $m>n$. }

Then \begin{itemize}
    \item $X_0=0$
    \item $\forall 0\leq s<t$: $X_t=X_s+\sum_{k=N_s+1}^{N_t}\xi_k$ 
\end{itemize}
which implies $X_t-X_s\stackrel{d}{=}\sum_{k=1}^{N_t-N_s}\xi_k\stackrel{d}{=}X_{t-s}$ and everything is independent, but 
$N_t-N_s=N_{t-s}$.
Since $N_t$ are càdlàc $\implies X_t$ is càdlàg. $(X_t)_{t\geq 0}$ is a Lévy process with 
\[\psi(\theta)=\lambda \int_{\R\setminus\{0\}}(1-e^{i\theta x}F(dx)).\]

Let $\Delta X_t\coloneqq X_t-X_{t^-}$ and for $B\in\mathcal{B}(\R\setminus\{0\})$,
\[\tilde{N}_t(B)\coloneqq \#\{s\in[0,t]:\Delta X_{s}\in B\}\]
the number of jumps of size $B$ in the time span $[0,t]$ and $t_k$ are the jump times.

$(X_t)_t$ is a Lévy process $\implies (\tilde{N}_t(B))$ has stationary independent increments.

$\tilde{N}_{t+s}(B)-\tilde{N}_(B)\stackrel{d}{=}\tilde{N}_t(B)$ for all $t,s\geq 0$ and $B$.

Therefore $(\tilde{N}_t(B))_t$ is a Lévy process with jumps of size $+1$, $\tilde{N}_0(B)=0$, which implies $\tilde{N}_t(B)$ is a poisson process.
The parameter of $\tilde{N}_t(B)$ is \[\bE(\tilde{N}_t(B))=\underbrace{t\cdot \lambda}_{=\bE(N_t)}\bP(\xi\in B)=t\underbrace{\lambda\int_B F(dx)}_{=\nu(dx)}=t\nu(B).\]
$\implies \nu(B)$ is the parameter of $\tilde{N}_t(B)$\footnote{As a process, not as a r.v.}.

\begin{aremark}
    Suppose we had a Levy process with finite measure (remove $ia\theta$ and $i\theta x1_{|x|<1}$). Then we can 
    normalize and define $F$ $\implies$ we get a compound poisson process!
\end{aremark}

Add a drift: 
\[X_t=\sum_{k=1}^{N_t}\xi_k+ct,t\geq 0\]

Then Lévy process of $X_t$ has characteristic exponent 
\[\psi(\theta)=\lambda\int (1-^{i\theta X})F(dx)\underline{-ic\theta}\]

\subsection{Brownian motion with drift}

Let $\mu_{s,\gamma}(dx)\coloneqq \frac{e^{-\frac{(x-\gamma)^2}{2s^2}}dx}{\sqrt{2\pi s^2}}, \text{ } s>0,\gamma\in\R$.

If $X\sim \mu_{s,\gamma}$, then 
\[\bE(e^{i\theta X})=e^{-\frac{1}{2}s^2\theta^2+i\theta \gamma}=\left(e^{-\frac{1}{2}\frac{s}{\sqrt{n}}^2\theta^2+i\theta \frac{\gamma}{n}}\right)^n\]
$\implies$ $X$ is infinitely divisible with characteristic exponent $\psi(\theta)$ with 
$\nu=0,\sigma=s^2,a=-\gamma$.
\[\implies \psi(\theta)=-i\theta\gamma +\frac{1}{2}s^2\theta^2.\]
and therefore \[X_t\coloneqq s B_t+\gamma t,\]
where $B_t$ is a standard BM, is the Lévy process with characteristic exponent $\psi(\theta)$.


