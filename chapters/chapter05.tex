\chapter{The Kardar-Parisi-Zhang (KPZ) equation}\beginlecture{18}{25.06.24}

\section{Heuristic derivation}

\dhighlight{Want:} describe the evolution of an interface \[x\mapsto h(x,t)\in\R\ x\in\R,t\geq 0\] where the growth is given by some local process

The analogue in $0$-dimensions would be \[h(t)\stackrel{t\in\Z}{=}\sum_{k=1}^t X_k,\] where the $X_k$ are iid r.v. with bounded variation. We know: 
\[\frac{h(t)}{t}\stackrel{\as}{\to}\mu\in\R (\text{law of large numbers})\] and 
\[\frac{h(t)-\mu t}{\sigma}t^{\frac{1}{2}}\stackrel{\bP}{\to}\mathcal{N}(0,1)\text{ (CLT)}\]
We call the first case \dhighlight{Macroscopic} and the second case \dhighlight{Fluctuations} or \dhighlight{mesoscopic}.

From a macroscopic point of view:
\[\implies \delta_t h =v(\partial_x h)\] for some function $v$ (Hamitten-Jacobi equation).

\dhighlight{Question:} How to add noise?

Consider some discrete examples

\dhighlight{Ballistic deposition:} $\forall x\in\Z$: independent Poisson process (intensity 1), $h(x,t)\in \Z,t\geq 0$

If at time $t$, there is a jump of the PP at site $x$, either 
\begin{enumerate}
    \item[(a)] add it on top of $h(x,t)$: $h(x,t)\to h(x,t)+1$
    \item[(b)] stick to one of the neighboring sites if they are higher: $h(x,t)\to\max\{h(x+1,t),h(x-1,t)\}$
    
\end{enumerate}We can use one Poisson process like previously for a finite window.

For the evolution:\[h(x,t)=\max\{h(x,t_-),h(x-1,t_-),h(x+1,t_-)\}\]

\dhighlight{Question:} How fast do the fluctuations grow?

Simulations shows 
\[\sqrt{\Var_x(h(x,t))}\approx t^{\frac{1}{3}}\text{ for large }t\]
If there would not be any interaction between sites $\stackrel{\text{CLT}}{\to}t^{\frac{1}{2}}\implies$ Stickiness reduces the fluctuations.

Influence of Stickiness is two-fold:
\begin{enumerate}
    \item[(a)] growth is depending on the slope (of the starting / boundary condition / the previous state?) $\to$ already modelled by $v(\partial_x h)$
    \item[(b)] some regularization going on $\to $ simplest regularization in the continuum $\to$ Heat equation:\[\partial_t h =\partial_x^2h\]
                $\implies \partial_t h=v(\partial_x h)+\partial_x^2 h$ 
\end{enumerate}

\dhighlight{Question:} What kind of noise should we consider?

Simplest would be to consider \highlight{Gaussian noise}, without correlation in space and time (like in the discrete model). 

\dhighlight{Informally:} the space-time white noise $\xi(x,t)$ is the centered \dhighlight{Gaussian process} with 
variance $\bE(\xi(x,t)\xi(y,t))=\delta(x-y)\delta(t-s)$

\begin{aremark}[WARNING]
    $\xi$ is not a function, but a distribution!
\end{aremark}

\begin{definition}\label{def:5.1}
    The \dhighlight{space time white noise $\xi$} is the Gaussian process satisfying 
    \[\forall \varphi_1(x,t),\varphi_2(x,t)\in C_0^\infty\]
    \marginnote{This is a bit informally wirtten, since $\xi$ is a distribution and not absolutely continuous w.r.t. Lebesgue}
    \begin{align*}
        \implies \bE\left(\int dxdt \varphi_1(x,t)\xi(x,t)\right)=0 \text{ centered}
    \end{align*}
    and 
    \[\bE\left(\int\varphi_1 (x,t)\xi(x,t)dxdt\cdot\int\varphi_2(y,s)\xi(y,s)dyds\right)=\int \varphi_1(x,t)\varphi_2(x,t)dxdt\]
\end{definition}

\begin{aremark}
    Careful: $\bE(\xi(x,t)^2)=\infty$. Pointwise this makes no sense!
\end{aremark}

$\implies$ We would get:
\[\partial_t h = \partial_x^2 h + v(\partial_x h)+\xi\]
Physicists argued: small $\partial_x h$
\begin{itemize}
    \item $v(\partial_x h)\approx v(0)+v'(0)\partial_x h + \frac{1}{2}v''(0)(\partial_x h)^2 +\dots$
    \item The first two terms can be removed wlog, since \[\partial_t h = v(0)\to h_0+v(0)t\implies h(x,t)+v(0)t\eqqcolon \tilde{h}(x,t)\]
        $\implies \partial_t\tilde{h} = \partial_x^2 \tilde{h}+ v'(0)\partial_x \tilde{h}+\dots$
    \item Physics should not depend on the reference system we use!    
\end{itemize}
Then \dhighlight{KPZ equation:}
\[\partial_t h = \partial_x^2 h +(\partial_x h)^2 +\xi\]
Or by keeping track of physical constants:
\[\partial_t h = \nu\partial_x^2 h +\lambda(\partial_x h)^2 +\sqrt{D}\xi\]

\dhighlight{Problem:} $|\partial_x h|$ is NOT SMALL!!!

Locally, $h$ is like a Brownian motion $\to \partial_x h$ is a distribution!
\[\int \varphi(x)\partial_x h(x)dx=-\int \varphi' h(x)dx\]
\[\int \varphi(x)(\partial_x h(x))^2dx\stackrel{????}{=}-\int \varphi' h(x)dx\]

\section{White noise}

\begin{example}\label{ex:5.2}
    Let $S$ be a nice subset of $\R^d$ and $e_1,\dots$ an orthonormal basis of $L^2(S)$.\marginnote{These might be eigenvectors of some operator}
    Let $\xi_1,\dots,$ iid $\mathcal{N}(0,1)$ r.v. 
    \[\implies\xi(x)=\sum_{n\geq 1}\xi_n e_n(x)\]
    is a white noise on $S$.
\end{example}

\begin{proof}
    Let $\varphi_1,\varphi_2:S\to\R$ be test functions.
    \begin{align*}
        \bE\left(\int_S \varphi_1\xi(x)dx\right)&=\bE\left(\int_S \varphi_1\sum_{n\geq 1}\xi_ne_n(x)dx\right)=\sum_{n\geq 1}\bE(\underbrace{\xi_n}_{\bE(\xi)=0}\int_S\varphi_1 e_n(x)dx=0)
    \end{align*}
    and 
    \begin{align*}
        \bE\left(\int_S\varphi_1(x) \xi(x)dx\int_S\varphi_2(y)\xi(y)dy\right)&=\sum_{n,m\geq 1}\bE(\xi_n\xi_m)\int_S\int_S \varphi_1(x)\varphi_2(y)e_n(x)e_m(y)dxdy\\
        &=\int dx dy \varphi_1(x)\varphi_2(y)\underbrace{\sum_{n\geq 1}e_n(x)e_m(x)}_{=1_{x=y}}=\int dx\varphi_1(x)\varphi_2(y).
    \end{align*}
\end{proof}

\begin{aremark}
    For orthonormal systems:
    \[f(x)=\sum_{n\geq 1} e_n(x)(e_n,f)=\int dy f(y)\underbrace{\sum_{n\geq 1} e_n(x)e_n(y)}_{=\delta(x-y)}\]
\end{aremark}

\begin{example}\label{ex:5.3}
    Let $S=\R_+$ and $\xi$ white noise on $S$. 
    \[\implies B(t)=\int_S 1_{[0,t)}(s)\xi(s)ds\]
    is a Brownian motion.
\end{example}


\beginlecture{19}{27.06.24}

\begin{lemma}\label{lem:5.4}
    Let $\xi$ be a white noise on $\R\times\R_+$. Then 
    \[\xi(x,t)\stackrel{d}{=}\epsilon^{\frac{z+1}{2}}\xi(\epsilon^{1}x,\epsilon^{z}t)\]
    where $z$ is called the \dhighlight{dynamical exponent} (in time).
\end{lemma}

\begin{proof}
    \begin{align*}
        \bE(\left(\int \varphi_1(x,t)\xi(x,t)dxdt\int \varphi_2(y,s)\xi(y,s)dyds\right))\stackrel{\text{def}}{=}\int \varphi_1(x,t)\varphi_2(x,t)dxdt 
    \end{align*}
    But also
    \begin{align*}
        &\bE\left(\int \phi_1(x,t)\epsilon^{\frac{z+1}{2}}\xi(\epsilon x,\epsilon^z t)dxdt\int \phi_2(y,s)\epsilon^{\frac{z+1}{2}}\xi(\epsilon y,\epsilon^z s)dyds  \right)\\
        &=\epsilon^{z+1}\bE\left(\varphi_1(\epsilon^{-1}x,\epsilon^{-z}t)\xi(x,t)dxdt\epsilon^{-(z+1)}\varphi_2(\epsilon^{-1}y,\epsilon^{-z}s)\xi(y,s)dyds\epsilon^{-(z+1)}\right)\\
        &=\epsilon^{-(z+1)}\int\varphi_1(\epsilon^{-1}x,\epsilon^{-z}t)\varphi_2(\epsilon^{-1}x,\epsilon^{-z}t)dxdt\\
        &=\int\varphi_1(x,t)\varphi_2(x,t)dxdt
    \end{align*}
    using $\tilde{x}=\epsilon x \mapsto x$ and $t=\epsilon^zt\mapsto t$.
\end{proof}

\section{The 1:2:3 scaling}
Start with \[\partial_t h =\nu\partial_x^2h+\lambda(\partial_x h)^2+ \sqrt{D}\xi.\]
Let us see that wlog we can set $\nu=\lambda=\frac{1}{2},D=1$. Define 
\[\tilde{h}(x,t)\coloneqq \alpha h(\gamma x,\beta t)\]
which implies 
\[\partial_t \tilde{h}(x,t)=\beta \alpha\dot{h}(\gamma x,\beta t)\]
using
\begin{itemize}
    \item $\partial_t  = \frac{1}{\alpha\beta}\partial_t \tilde{h}$
    \item $\partial_x h = \frac{1}{\alpha\gamma}\partial_x\tilde{h}$
    \item $\partial_x^2 h = \frac{1}{\alpha\gamma^2}\partial_x^2\tilde{h}$
\end{itemize}
we get 
\begin{align*}
    \partial_t \tilde{h} &= \alpha\beta\partial_t h \\
    &=\alpha\beta\left[\nu\frac{1}{\alpha\gamma^2}\partial_x^2\tilde{h}+\lambda\frac{1}{\alpha^2\gamma^2}(\partial_x \tilde{h})^2 +\sqrt{D}\frac{1}{\sqrt{\beta\gamma}}\xi\right]\\
    &=\underbrace{\frac{\beta \nu}{\gamma^2}}_{=\frac{1}{2}}\partial_x^2\tilde{h}+\underbrace{\frac{\lambda\beta}{\alpha\gamma^2}}_{=\frac{1}{2}}(\partial_x \tilde{h})^2+\underbrace{\sqrt{D}\alpha\sqrt{\frac{\beta}{\gamma}}}_{=1}\xi 
\end{align*}

which implies $\alpha=\frac{\lambda}{\nu},\beta=\frac{2\nu^5}{D^2\lambda^4},\gamma=\frac{2\nu^3}{D\lambda^2}$.
The standard choice we will use below is 
\[\partial_t h = \frac{1}{2}\partial_x^2 h + \frac{1}{2}(\partial_x h)^2+\xi\]
which is the \dhighlight{KPZ equation}.

Start with $\partial_t h = \frac{1}{2}\partial_x^2 h +\frac{1}{2}(\partial_x h)^2+\xi$. Define 
\[h_\epsilon(x,t)\coloneqq \underbrace{\epsilon^b}_{T^{-b/z}x} h(\underbrace{\epsilon^{-1}x}_{x T^{1/z}},\underbrace{\epsilon^{-z}t}_{\eqqcolon tT})\]
\[h_\epsilon(x,1)\coloneqq \epsilon^b h(\epsilon^{-1}x,\epsilon^{-z}1)\]
If $b,z$ choosen s.t. $\lim_{\epsilon\downarrow 0} h_\epsilon(x,1)$ is well defined ( and non-trivial), then 
the Fluctuations of $h$ at large time $T$ is $O(T^{b/z})$ and correlations are non-trivial over distances $O(T^\frac{1}{z})$.

Discrete model: Picture here \dots 

If $(p-q)T^c\to 0$ the height converges to the height of KPZ eq.

Stationary measure for the increment are simple random walks. 

Under scaling: One gets BM increments. If one starts with a 2-sided BM, more precisely
\[h(x,0)=\sqrt{2}B(x)\implies h(x,t)-h(0,t)\stackrel{d}{=}\sqrt{2}B(x)\]

And therefore $b=\frac{1}{2}$.

\begin{itemize}
    \item $\partial_t h = \epsilon^{z-b}\partial_t h_\epsilon$
    \item $\partial_x h=\epsilon^{1-b}\partial_t h_\epsilon$
    \item $\partial_x^2 h = \epsilon^{2-b}\partial_x^2 h_\epsilon$
    \item $xi(x,\epsilon^{-1},t\epsilon^{-z})\stackrel{d}{=}\epsilon^{\frac{z+1}{2}}\xi(x,t)$
\end{itemize}

$h_\epsilon$ solves
\begin{align*}
    \implies \partial_t h_\epsilon &= \frac{1}{2}\epsilon^{2-z}\partial_x h_\epsilon + \frac{1}{2}\epsilon^{2-z-b}(\partial_x h_\epsilon)^2 + \epsilon^{b-\frac{z+1}{2}+\frac{1}{2}}\xi\\
    &\stackrel{b=\frac{1}{2}}{=}\frac{1}{2}\epsilon^{2-z}\partial_x^2 h_\epsilon + \frac{1}{2}\epsilon^{\frac{3}{2}-z}(\partial_x h_\epsilon)^2+\epsilon^{1-\frac{z}{2}}\xi
\end{align*}

The nonlinear term $\frac{1}{2}\epsilon^{\frac{3}{2}-z}(\partial_x h_\epsilon)^2$ is still relevant, but not $\to\infty$. This gives the idea 
$z=\frac{3}{2}$.

\[\partial_t h_\epsilon = \frac{1}{2}\epsilon^{\frac{1}{2}}\partial_x^2 h_\epsilon + \frac{1}{2}(\partial_x h_\epsilon)^2 + \epsilon^{\frac{1}{4}}\xi\]

$\implies$ at large time $t$, fluctuations of $h(x,t)\approx t^{\frac{1}{3}}$, correlations in space $\approx t^\frac{2}{3}$.

\dhighlight{Question: Are the $\epsilon^{\frac{1}{2}}$ and $\epsilon^{\frac{1}{4}}$ terms relevant as $\epsilon\to 0$?}

1.: $\partial_t h_\epsilon = \frac{1}{2}(\partial_x h_\epsilon)^2$ is not well-posed.
with the term $\sqrt{\epsilon}\partial_x^2 h_\epsilon$
\[\implies \partial_t h = \frac{1}{2}(\partial_x h)^2+\frac{1}{2}\sqrt{\epsilon}\partial_x^2h\]
which is the viscocity solution. Then taking take $\epsilon\to 0$ gives \[h(x,t)=\sup_{y}(h(y,0)-\frac{(x-y)^2}{2t})\]

If you start with $h_\epsilon(x,0)=\sqrt{2}B(x)\implies h(x,t)$ has some explicit expression. This solution 
do not satisfy $h(x,t)-h(x,0)\stackrel{d}{=}\sqrt{2}B(x)$. Therefore $\epsilon^{\frac{1}{4}}\xi$ is relevant as $\epsilon\to 0$.

\section{The linearized case: Edwards-Wilkinsen equation}

Let $\lambda=0\implies$

\begin{equation}\label{Edwards-Wilkinsen}
    \begin{cases}
        \partial_t h  &= \frac{1}{2}\partial_x^2 h+\xi\\
        h(x,0)&=h_0(x)
    \end{cases}
\end{equation}
It is linear!
\begin{remark}
    \[\begin{cases}
        \frac{\partial}{\partial t}h &= Ah +f\\
        h(0)&=h_0
    \end{cases}\implies\]
    denote by $\tilde{h}(t)\coloneqq e^{At}h_0$ as the solution with $f=0$. The variation of constant formula gives:
    \[h(t)=e^{At}h_0+\int_0^te^{A(t-s)}f(s)ds\]

Without $\Xi$: $h(x,t)=\int_\R P_t(x-y)h_0(y)dy,P_t(x)=\frac{e^{-\frac{x^2}{2t}}}{\sqrt{2\pi t}}$
\[\implies h(x,t)=\int_\R P_t(x-y)h_0(y)dy+\int_0^t\int_\R P_{t-s}(x-y)\xi(y,s)dyds\]
\end{remark}

Scaling of EW:

\[h_\epsilon(x,t)\coloneqq \epsilon^{\frac{1}{2}}h(\epsilon^{-1}x,\epsilon^{-z}t)=\epsilon^{\frac{1}{2}}h(\epsilon^{-1}x,\epsilon^{-2}t)\]
yields ( with the same calculations as before, which leads to $z=2$) 
\[\partial_t h_\epsilon = \frac{1}{2}\epsilon^{2-z}\partial_x^2 h_\epsilon+\epsilon^{1-\frac{z}{2}}\xi=\frac{1}{2}\partial_x^2 h_\epsilon+\xi\]
Therefore, at large time $t$, fluctuations will be $O(t^{\frac{b}{z}})=O(t^{\frac{1}{4}})$. The nonlinear term in the KPZ increases the size of the fluctuations!

\section{The Cole-Hopf transformation}\marginnote{This sadly only works in $1$ dimension}
We have: $\partial_t h=\frac{1}{2}\partial_x^2 h +\frac{1}{2}(\partial_x h )^2+\xi$

\begin{lemma}\label{lem:5.5}
    Assume that $h$ solves
    \[\partial_t h = \frac{1}{2}\partial_x^2h + \frac{1}{2}(\partial_x h)^2+V\]
    with $V$ some nice $V$ in $(x,t)$.

    Then $z=e^h$ solves 
    \[\partial_t z = \frac{1}{2}\partial_x^2 z + zV\]
\end{lemma}

\begin{proof}
    $h=\ln(z)$, and therefore 
    \begin{align*}
        \frac{1}{2}\frac{(\partial_x z)^2}{z^2}+\frac{1}{2}\left(\frac{\partial_x^2 z}{z}-\frac{(\partial_x z)^2}{z^2}\right)+V=\partial_t h= \frac{1}{z}\partial_t z
    \end{align*}
\end{proof}

Suppose that we regularize $\xi$ in space:
\[\xi\to\xi_\epsilon\coloneqq \int dy \rho_\epsilon(x-y)\xi(y,t)\]
where $\rho_\epsilon=\frac{e^{\frac{-x^2}{2\epsilon}}}{\sqrt{2\pi \epsilon}}$.

Solve $\partial_t z = \frac{1}{2}\partial_x^2 z + z\xi_\epsilon$. Set $h(x,t)\coloneqq \ln(z(x,t)).$

%TODO_ Number
Exercise: In the above setting, h solves:
\[\partial_t h = \frac{1}{2}\partial_x^2 h + \left(\frac{1}{2}(\partial_x h)^2-\frac{C}{\epsilon}\right)+\xi_\epsilon\]
for some explicit $C$.

If one solves $\partial_t z = \frac{1}{2}\partial_x^2 z + z\xi$ and define $h(x,t)\coloneqq \ln(z(x,t))$.
This is called the \dhighlight{Cole-Hopf solution of the KPZ eq.}.

\beginlecture{20}{02.07.24}

Exam:
\begin{itemize}
    \item Speak about topic we choose (freely): 15 min discussion
    \item He will not talk about us about the last part 
    \item Talk the rest of the time about the rest of the lecture
\end{itemize}

If we are smooth enough, the $C$ in the Cole-Hopf-equation will be $0$.

In order to have a limit for $\epsilon\to0$:
\[h(x,t)\coloneqq \tilde{h}(x,t)-\frac{c}{\epsilon}t\]
$\xi_\epsilon$ is still centered Gaussian with covariance:
\begin{align*}
    \bE(\xi_\epsilon(x,t)\xi_\epsilon(y,s))&=\bE\left(\int_\R d\tilde{x}\rho_\epsilon(x-\tilde{x})\xi(\tilde{x},t)\int_\R d\tilde{y}\rho_\epsilon(y-\tilde{y})\xi(\tilde{y},s)\right)\\
    &=\delta(t-s)\cdot\underbrace{\int_{R}d\tilde{x}\rho_\epsilon(x-\tilde{x})\rho_\epsilon(y-\tilde{x})}_{\eqqcolon C_\epsilon(x-y)}
\end{align*}

In particular, $C_\epsilon(0)=\frac{1}{\sqrt{2\pi}\epsilon}=\frac{c}{\epsilon}$.

Let $z=^\epsilon$ be the solution of 
\[(\star)\ \partial_tz=\frac{1}{2}\partial_x^2 z+ z\cdot \xi_\epsilon\]
\begin{itemize}
    \item One checks $z^\epsilon\stackrel{\epsilon\to0}{\to}z$ uniformly on compact sets.
    \item $z(x,t)>0$ (starting from $z(x,0)>0$), then define $h(x,t)=\ln(z(x,t))$
    \item Define $h^\epsilon\coloneq \ln z^\epsilon(x,t)$. it solves the following equation:\[\partial_x h^\epsilon=\frac{1}{2}\partial_x^2h^\epsilon+\frac{1}{2}(\partial_x h^\epsilon)^2-C_\epsilon(0)+\xi_\epsilon\]
    \item In differential notation \[d z^\epsilon = \frac{1}{2}\partial_x^2 z^\epsilon dt+z^\epsilon \underbrace{\xi_\epsilon dt}_{=d\xi_\epsilon}\]
\end{itemize}

Therefore \begin{align*}
    dh^\epsilon&=d\ln z^\epsilon\stackrel{\text{Itô}}{=}\frac{1}{z^\epsilon}dz^\epsilon-\frac{1}{2}\frac{1}{(z^\epsilon)^2}(dz^\epsilon)^2\\
    &=\frac{1}{2}\frac{1}{z^\epsilon}\partial_x^2z^\epsilon dt +d\xi_\epsilon-\frac{1}{2}(d\xi_\epsilon)^2\\
    &=\frac{1}{2}\partial_x^2 h^\epsilon dt + \frac{1}{2}(\partial_x h^\epsilon)^2dt + d\xi_\epsilon-\frac{1}{2}C_\epsilon(0) dt
\end{align*}

which implies 
\[\partial_t h^\epsilon = \frac{1}{2}\partial_x^2 h^\epsilon+\frac{1}{\epsilon}\left((\partial_x h^\epsilon)^2-C_\epsilon(0)\right)+\xi_\epsilon\]

The Cole-Hopf solution of the KPZ equation is 
\[h(x,t)=\ln(z(x,t))\]
where $z$ solves \[\begin{cases}
    \partial_t z &= \frac{1}{2}\partial_x^2 z +z\cdot\xi \\
    z(x,0)&=e^{h(x,0)}
\end{cases}\]

The difficulty in the multiplicative \dhighlight{stochastic heat equation (SHE)} is the term $z\xi$.
Still it is much better than squaring a distribution.

Since $h(x,t)$ will be locally BM, $z(x,t)=e^{h(x,t)}$ is also locally gaussian. In the integral form:
\[z\xi\to \int_0^t z\underbrace{\xi ds}_{\approx dB_s}\]

\subsection{Mild solutions}

\begin{definition}\label{def:5.7}
    Let $\begin{cases}
        \partial_t z&=\frac{1}{2}\partial_x^2 z + z\cdot \xi \\
        z(x,0)&=z_0(x)
    \end{cases}$ be $(\star)$. $z$ is a \dhighlight{mild solution} of $(\star)$, if 
    \[z(x,t)=\int_R P_t (x-y)z_0(y)dy+\int_0^tds\int_\R dy P_{t-s}(x-y)z(y,s)\xi(y,s)\ (A)\]
    where $z$ is predictable and 
    \[\int_0^tds\int_\R (P_{t-s}(x-y))^2\bE(z(y,s))^2<\infty\ (B)\]
\end{definition}

\begin{theorem}\label{thm:5.8}
    Suppose that $\sup_{x\in\R}\bE(z_0^2(x))<\infty$. Then there exists a unique, progressively measurable $z(x,t)$ satisfying 
    (A) and (B).
\end{theorem}

\begin{proof}
    Let $z^0(x,t)=0$. Define iteratively $z^{n+1}(x,t)=\int_\R P_t(x-y)z_0(y)dy+\int_0^t ds \int_\R P_{t-s}(x-y)z^n(y,s)\xi(y,s)$.
    Let $\delta z^{n}(x,t)=z^{n+1}(x,t)-z^n(x,t)$.
    \begin{align*}
        \delta z^{n+1}=\int_0^t ds\int_\R dy P_{t-s}(x-y)\delta z^m(y,s)\xi(y,s)
    \end{align*}

    \begin{align*}
        \bE\left(|\delta z^{n+1}|^2\right) = \int_0^tds\int_\R dy (P_{t-s}(x-y))^2\bE(\vert \delta z^n(y,s)\vert^2).
    \end{align*}
    Define $f^n(t)\coloneqq \sup_{\stackrel{x\in\R}{0\leq s\leq t}}\bE(\vert \delta z^n(x,s)\vert^2)$. Assume that $f^n(t)<\infty$ 
    
    For $n=0$ it is true: $\delta z^1(x,t)=z^1(x,t)=\int_\R P_t(x-y)z_0(y)dy$ and therefore 
    \[\bE\left(\left\vert \int_\R P_t(x-y)z_0(y)\right\vert^2\right)\stackrel{\text{Jensen}}{\leq}\bE\left(\int_\R P_t(x-y)|z_0(y)|^2dy\right)=\int_\R P_t(x-y)\bE(|z_0(y)|^2)dy\leq \sup_{y\in\R}\bE(|z_0(y|^2)\]
    Now we bound $f^{n+1}$:
    \begin{align*}
        f^{n+1}(t)&=\sup_{\stackrel{x\in\R}{0\leq s \leq t}}\int_0^sd\tilde{s} \int_\R dy (P_{t-\tilde{s}}(x-y))^2 f^n(\tilde{s})\\
        &\leq \sup_{0\leq s\leq t}\int_0^s d\tilde{s} f^n(\tilde{s})\frac{C}{\sqrt{s-\tilde{s}}}\\
        &\leq \int_0^t d\tilde{s}\frac{f^n(\tilde{s})}{\sqrt{t-\tilde{s}}}C_2\leq \tilde{C}\int d\tilde{s}\int_0^{\tilde{s}}du \frac{f^{n-1}(u)}{\sqrt{t-\tilde{s}}\sqrt{\tilde{s}-u}}\\
        &=C_3 \int_0^t du f^{n-1}(u)\underbrace{\int_u^t d\tilde{s}\frac{1}{\sqrt{\tilde{s}-u}}\sqrt{t-\tilde{s}}}_{\leq C_4}\\
        &\implies f^{n+1}(t)\leq \hat{C}\int_0^t du f^{n-1}(u)\leq (\hat{C})^2\int_0^t du \int_0^ud u -f^{n-3}(u)\\
        &\implies f^{2n}(t)\leq (\hat{C})^n \frac{t^n}{n!}\sup_{y\in\R} \bE(|z_0(y)|^2) \stackrel{n\to\infty}{\to} 0
    \end{align*}
    where the convergence happens very fast! There there exists a solution $z$ satisfying (A) and (B)
\end{proof}

For $z_0=e^{h_0(x)},h_0(x)=B(x)1_{x\geq 0}$, we can not use \ref{thm:5.8}! Our result is a bit to stiff.

\begin{lemma}\label{lem:5.9}
    Let $z(x,t)$ satisfy (A) and (B) and 
    \[\bE\left(\left\vert\sup_{A\subseteq [-n,n]}\int_A z_0(x)dx\right\vert\right)\leq ce^{cn}\text{ for some }c\in(0,\infty)\]
    Then there exists $\tilde{c}<\infty$, $\bE(z^2(x,t))\leq \frac{\tilde{c}}{\sqrt{t}}e^{\tilde{c}(t+|x|)}$ for all $x\in\R,t>0$.

\end{lemma}

   In the special case $z_0(x)=\delta_0(x)$, there exists $c=c(T)$ s.t.$\bE(z^2(x,t))\leq c(P_t(x))^2$ for all $0<t\leq T$, where $P_t=\frac{1}{\sqrt{2\pi t}}e^{\frac{-x^2}{2t}}$

We want to know: What is $\bP(z(x,t)\leq s)$ for a fixed $x$?

\beginlecture{21}{04.07.24}

\begin{align*}
    \bE\left(\left\vert \int_0^t X_s\underbrace{\xi_sds}_{dB_s}\right\vert^2\right)&=\bE\left(\int_0^t X_s^2ds\right)\\
    &=\int_0^t\bE(X_s^2)ds
\end{align*}


\begin{align*}
    \bE\left(\left\vert \int_0^t ds \int dy p_{t-s}(x-y)\delta z^n(y,s)\xi(y,s)\right\vert^2\right)&\stackrel{?}{=}\int_0^t ds \int dy (P_{t-s}(x-y))^2\bE(\left\vert \delta z^n(y,s)\right\vert^2)
\end{align*}

And a correction to the iteration from last time:
\[f^n(t)\sup_{x}\sup_{0\leq s\leq t}\bE\left(\vert \delta z^n(x,s)\vert^2\right)\]
instead consider 
\[g_n(t)=\sup_{x}\bE\left(\vert \delta z^n(x,s)\vert^2\right)\]

\dhighlight{End of correction}


\dhighlight{Question: $\bP(h(0,t)\leq s)=?$}

\section{Feynman-Kac representation}

\begin{equation}
    \begin{cases}
        u(x,0) &= f(x)\\
        \partial_t u=\frac{1}{2}\partial_x^2 u + V(x,t)u
    \end{cases}
\end{equation}

If $V$ is nice enough (e.g. bounded) $\implies$ (see lecture 20) a solution is 
\[u(x,t)=\bE_x\left(f(B_t)e^{\int_0^t V(t-s,B_s)}ds\right)\]
for some BM $(B_t)_{t\geq 0}$starting in $x$ and $\tilde{B}$ running backwards with $B_t=x$:
\begin{align*}
    u(x,t)&=\bE_x\left(f(\tilde{B}_0)e^{\int_0^t V(s,\tilde{B}_s)}ds\right)
\end{align*}
\dhighlight{Formally:} $z(x,t)=\bE_x(z(0,B_0))\cdot e^{\int_0^t \xi(s,B_s)ds}$ with $B$ BM from time $t$ to time $0$ with $B_t=x$.

Issue of this: 
\begin{align*}
    e^{\int_0^t\xi(s,B_s)ds}&)\sum_{n\geq 0}\frac{1}{N!}\int_0^tds_1\dots,\int_0^t ds_n\xi(s_1,B_{s_1})\dots\xi(s_n,B_{s_n})
\end{align*}

We have to be careful about the diagonal, to avoid squaring the distribution $\xi(s,B_s)$.


\begin{align*}
    e^{\int_0^t\xi(s,B_s)ds}&=\sum_{n\geq 0}\frac{1}{n!}\int_0^tds_1\dots\int_0^t ds_n\xi(s_1,B_{s_1})\dots\xi(s_n,B_{s_n})\\
    &=\sum_{n\geq 0}\frac{1}{n!}\int_{0<s_1<\dots<s_n<t}d_{s_1}\dots ds_n\xi(s_1,B_{s_1})\dots\xi(s_n,B_{s_n})
\end{align*}

\begin{align*}
    \implies z(x,t)=\sum_{n\geq 0}\int_\R dy \int_{0<s_1<\dots< s_n<t}d_{s_1}\dots d_{s_n}\prod_{k=1}^N \int dx_k\xi(x_k,s_k)P_{x,t}(y,0,x_1,s_1,\dots,x_n,s_n)z(y,0)
\end{align*}

\subsection{A discrete example}

Take for example $\Z^2$ as time $\times $ space. Starting in $(x,t)$ 
look at a random walk with path $\pi$ with nearest neighbor steps. To each $(i,j)\in \Z^2$-> iid random variable $\omega(i,j)$
\[\implies Z(x,n)\coloneqq \bE\left(z(0,x_0)e^{-\beta\sum_{k=1}^n \omega(k,x_k)}\right)\]

We can imagine many other discretizations.

\section{Moments of the multiplicative SHE}

\[\bE(e^{-\lambda z(x,t)})=?\]
if $z(x,t)\geq 0$,, the laplace transform above is well defined.
\[=\bE\left(\sum_{n\geq 0}\frac{(-\lambda)^2}{n!}z(x,t)^n\right)\stackrel{?}{=}\sum_{n\geq 0}\frac{(-\lambda)^n}{n!}\bE(z(x,t)^n)\]
which is useful, since $\bE(z(x,t)^n)$ has explicit expressions:

\dhighlight{Strategy to compute these moments:} Start with noise $\xi_\epsilon=\rho_{\epsilon}*\xi$, with $\rho_\epsilon(x,t)=\epsilon^{-1}\rho(\epsilon^{0.5}x,\epsilon^{0.5}t)$, where $\rho$ is even, smooth, compactly supported non-negative function with
\[\int\rho(x,t)dxdt=1.\]

Then $z^\epsilon(x,t)=\bE_x\left(z(0,B_0)e^{\int_0^txi_\epsilon(s,B_s)ds}e^{-C_\epsilon t}\right)$ and 
\[z^\epsilon(x,t)\to z(x,t)\text{ nicely}\]
As $\epsilon\to 0$
\[\bE((z(x,t))^n)=\bE_{(x,\dots,x)}\left(e^{-\frac{1}{2}\sum_{i<j}\int_0^t\sgn(B_i-B_j)(dB_i-dB_j)}F(B(t))\right)\]
for some $n$-dimensional BM, $B_i(0)=x$ and 
\[F(B)=e^{\frac{1}{2}\sum_{i<j}|B_i-B_j|}\prod_{i=1}^nz_0(B_i)\]
Using antisymmetry:
\begin{align*}
    \sum_{i<j}\int_0^t\sgn(B_i-B_j)(dB_i-dB_j)&=\sum_{i=1}^N\sum_{j\neq i}\int_0^t\sgn(B_i-B_j)\underbrace{dB_i}_{=dB_t^i}\\
    \bE(e^{-\frac{1}{2}\sum_{i=1}^n\left(\sum_{j\neq i} \int_0^t \sgn(B_i-B_j)dB_i\right)})&=\bE\left(e^{\frac{1}{8}\sum_{i=1}^n\int_0^t\sum_{j\neq i}\underbrace{(\sgn(B_i-B_j)^2)}_{=1}}\right)ds
\end{align*} 
Trick:
\[\sum_{i=1}^n\left(\sum_{j\neq i}\sgn(B_i-B_j)\right)^2=\frac{1}{3}n(n^2-1)\]

$\implies \bE(z^n(x,t))\approx e^{\frac{n^3}{24}}$. We then get 
\begin{align*}
    \sum_{n\geq 0}\frac{(-\lambda)^n}{n!}\underbrace{\bE\left(z(x,t)^n\right)}_{\approx e^{\frac{1}{24}n^3}}\to \text{not ok}
\end{align*}
$\implies$ The moments grow to fats and they do not determine the distribution of $z(x,t)$.
there is some $f$:
\[e^{t^3}=\int dz e^{zt}f(z)\]

\section{The q-TASEP}

The solution of the multiplicative SHE with $z_0(x)=\delta_0(x)$ is 
\[z(x,t)=\bE_x\left(   \delta_0(B_0)e^{\int_0^t \xi(s,B_s)ds}\right)\]
Fix $q\in(0,1)$.

\dhighlight{TASEP=Totally Asymmetric Exclusion Process}

\begin{itemize}
    \item \highlight{Configuration space:} particle $n$ at time $t$ is at position $x_n(t)$ with $x_{n+1}(t)<x_n(t)$ for all $t,n$.
    \item \highlight{Dynamics:} each particle tries to jump to its right by one ($+1$) with \highlight{rate} 1. The jump takes place only if the arrival site is empty.
\end{itemize}

\subsection{q-TASEP}

here the jump rate of particle $n$ is given by \[1-q^{X_{n-1}(t)-X_n(t)-1}.\]

Consider the \highlight{initial condition} $X_n(0)=-n,n\geq 1$.

\dhighlight{Height function:}
\[\begin{cases}
    h(0,0)=0&h(0,t)=2(\text{\# particles which jumped from  to  during }[0,t])\\
    h(x+1,t)-h(x,t)&=1 
\end{cases}\]

\[=\begin{cases}
    h(0,0)&=h(0,t)=2(\text{\# particles which jumped from  to  during }[0,t])\\
    \begin{cases}
        1 & \not\exists \text{ particle at time }t\\ 
        -1 & \exists \text{ particle at time }t
\end{cases}\end{cases}
\]

One can compute: (using Macdonald polynomials)
\begin{align*}
    \bE\left(q^{l\cdot \underbrace{(x_n(t)+n)}_{\lambda_n(t=x_n(t)-n)}}\right)
\end{align*} 
$x_n(0)=-n\implies x_n(t)+n\geq 0$
\[0\leq q^(l\cdot \lambda_n(t))\leq n\]

\dhighlight{Introduce:} $(x;q)_\infty\coloneqq \prod_{k\geq 0}(1-x^k)$

\dhighlight{Generating function}\[\bE\left(\frac{1}{(\zeta q^{\lambda_n(t)};q)_\infty}\right)\stackrel{\text{Re}\zeta<0}{=}\bE\left(\sum_{l\geq 0}\frac{q^{l\cdot \lambda_n(t)\xi^l}}{(1-q)\cdot\dots\cdot (1-q^l)}\right)\stackrel{|\zeta|<1}{=}\sum_{l\geq 0}\frac{\zeta^l}{(1-q)\cdot\dots\cdot (1-q^l)}\bE(q^{l\cdot \lambda_n(t)})\]

\highlight{magically} (algebra is the magician)
\begin{align*}
    \sum_{l\geq 0}\frac{\zeta^l}{(1-q)\cdot\dots\cdot (1-q^l)}\bE(q^{l\cdot \lambda_n(t)})&=\det(1+K_{\zeta,n})_{L^2(\mathbb{C})}
\end{align*}
where $K_\zeta$ is an operator and we use the Fredholm determinant
\begin{align*}
    \coloneqq \sum_{m\geq 0}\frac{1}{m!}\int_{\mathbb{C}}dz_1\dots\int_{\mathbb{C}}d z_n\det[K_{\zeta,n}(z_i,z_j)]_{1\leq i,j\leq m}
\end{align*}
which is \highlight{purely analytic}, there is no randomness involved!

Next steps:

$q\to 1$ limit will lead to a semi-discrete polymov model $\to$ solution $z(x,t)$


\begin{aremark}
    We can sometimes drop the condition $|\zeta|<1$ if the rhs is well defined, since we have to analytic functions which agree on an open set.
\end{aremark}

\beginlecture{22}{09.07.24}
I was ill, so no notes :/
\newpage
\beginlecture{23}{11.07.24}
\section{Models and asymptotics for KPZ}
Presentation

